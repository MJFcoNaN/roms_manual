\section{Subversion}
\label{Svn}

The ROMS source code is distributed using
\href{http://subversion.tigris.org}{Subversion} (svn). There are svn
clients available for nearly every operating system and many
resources available, including an O'Reilly book \citep{SVN}. I'll
just cover a few basics here, taken in part from the
\href{https://www.myroms.org/wiki/index.php/Subversion}{ROMS wiki}.

If you wish to maintain your own copy of ROMS in a source code
repository, you may want to investigate \href{http://git-scm.com/}{git},
which has the ability to download from an svn server. It too has
an O'Reilly book \citep{GIT2}, plus I enjoyed
\citet{GIT1}. I have rambled at length about \code{git}
on the \href{https://www.myroms.org/wiki/index.php/ROMS_git}{ROMS
wiki} because I truly believe that \code{svn} is not the best tool
for most ROMS users (those who can't check back in). My ramblings
about the shortcomings of \code{svn} have driven one ROMS user to adopt
\href{http://mercurial.selenic.com/}{Mercurial} instead, another
valid option.

\subsection{Overview}
Subversion is a tool for managing software development that keeps
track of who modified what and allows the return to a previous
version if changes don't work as expected. All the ROMS/TOMS files
are stored in a \code{svn} repository on www.myroms.org with access
controlled by requiring authentication with the same ROMS
Username/Password combination assigned to registered users of the
ROMS Forum.

This svn repository is the official version of the code which only
the developers are allowed to change. Users should download the ROMS
code to their local machines using an svn client. Don't attempt to
use a regular web browser to browse or download files from the svn
repository - there are much better tools for interacting with the
code repository.

We strongly recommend that users always check out the current {\em trunk}
version since this has the most recent updates and bug fixes. The
tags version is kept largely as an historical record of stable
releases at the conclusion of major code upgrades.

Below is a general description of how subversion works. Please look
at the svn book \citep{SVN} for more detailed information and the wiki
for more on some available GUI clients.

\subsection{Checking out the code}
WARNING: It is strongly suggested that you checkout the ROMS source
code using the same operating system you wish to compile and run
ROMS on. If you download the code on a Windows machine and wish to
run it on a non-Windows machine you will need convert the line
endings with a utility like dos2unix or recode. Even with these
utilities you may still have problems compiling ROMS.

In order download source code from a Subversion repository, the svn
client software must be installed on your local machine. If you are
compiling subversion on your own be sure to build it with SSL
support or you will not be able to download the ROMS source code.
Most Linux distributions come with subversion (the command name is
svn), so shell commands may be used without installing additional
software. If your username on your local computer is not the same as
your ROMS username you will need to pass the --username <username>
option to svn; an example is given below. The general form of
subversion commands is: 
\begin{verbatim}
     svn action from to {optional_qualifiers} 
\end{verbatim}
To check-out the files from the ROMS repository trunk:
\begin{verbatim}
    svn checkout https://www.myroms.org/svn/src/trunk MyDir
\end{verbatim}
where MyDir is the destination directory on your local computer.
Note the https rather than http. If your username on your local
computer is not the same as your ROMS username you will need to pass
the --username option to svn:
\begin{verbatim}
   svn checkout --username joe_roms https://www.myroms.org/svn/src/trunk MyDir
\end{verbatim}
You only check out once, after that, a hidden directory called .svn
exists to keep track of the source, destination and a bunch of other
information. Your username and password will also be saved.

\subsection{Updates}
Now and again, you might feel the urge to get up to speed with the
latest changes that have been made to the ROMS repository. When that
happens, simply go to the directory that was ``MyDir'' above and
type:
\begin{verbatim}
    svn update
\end{verbatim}
Subversion will remember where you checked out from before and
see if a newer revision exists. If so, it will download and apply
all the relevant changes.

\subsection{Code changes}
As you use ROMS, you may find yourself adding new files and new chunks of
code to existing files. Unless you are a developer with write access to the
repository at www.myroms.org, you have no easy way to save your changes
within the svn framework, since any one directory can only point to one
repository. To keep getting updates from the trunk, you must keep using the
svn server at myroms.org. At the very least, saving a tarball before
fetching major updates is a prudent step.

\subsection{Conflicts}
Sometimes when you make changes to your copy of the ROMS code, those changes
will conflict with changes made to the repository. This means that the
changes from the server overlapped with your own, and now you have to
manually choose between them.

Whenever a conflict occurs, three things typically occur to assist you in
resolving that conflict:
\begin{itemize}
        \item Subversion halts during the update, offering you several choices,
and remembers that the file is in a state of conflict if you don't clear it
right then.
        \item If Subversion considers the file to be mergeable, it places
conflict markers (special strings of text which delimit the ``sides'' of the
conflict, usually ``<'' and ``>'' characters) into the file to visibly
demonstrate the overlapping areas.
        \item For every conflicted file, Subversion places three extra
unversioned (not under Subversion control) files in your working copy: 
\begin{klist}
            \kitem{filename.mine}: This is your file as it existed
      in your working copy (local copy) before you updated your working
      copy. This file has only your latest changes in it. (If Subversion
      considers the file to be unmergeable, then the .mine file isn't
      created, since it would be identical to the working file.)
            \kitem{filename.rOLDREV}: This is the file that was the
      BASE revision before you updated your working copy. That is,
      the file that you checked out before you made your latest edits.
            \kitem{filename.rNEWREV}: This is the file that your
      Subversion client just received from the server when you updated
      your working copy. This file corresponds to the HEAD (latest)
      revision of the repository.
  \end{klist}
\end{itemize}
For example, let's say you checked out revision 280 and made some changes
to a file, for instance \code{User/Functionals/ana\_hmixcoef.h}.
Now you want to update your
ROMS source code to take advantage of a new algorithm but when you run
svn update your \code{ana\_hmixcoef.h} is now in conflict with the new
version in the repository:
\begin{verbatim}
    > svn update
    U Version
    U ROMS/Modules/mod_mixing.F
    U ROMS/Functionals/ana_hmixcoef.h
    C User/Functionals/ana_hmixcoef.h
    Updated to revision 291.

    >
\end{verbatim}
The above is with an older version of svn. The latest, greatest does this:
\begin{verbatim}
    Conflict discovered in 'ROMS/Utility/inp_par.F'.
    Select: (p) postpone, (df) diff-full, (e) edit,
    (mc) mine-conflict, (tc) theirs-conflict,
    (s) show all options:
\end{verbatim}
Selecting (p) will behave as the old version.

If you get a conflict, you need to do one of three things:
\begin{itemize}
        \item Merge the conflicted text ``by hand'' by examining and
     editing the conflict markers within the file.
        \item Copy one of the temporary files on top of your working file.
        \item Run svn revert <filename> to throw away all of your local
     changes.
\end{itemize}
Once you've resolved the conflict, you need to let Subversion know by
running ``svn resolved''. This removes the three temporary files and
Subversion no longer considers the file to be in a state of conflict. More
on this below.

\subsubsection{Merging conflicts by hand}
To merge your changes with those from the latest revision in the repository,
open \code{ana\_hmixcoef.h} in your favorite editor and
look for a string of ``<'' characters. You should see something like this:
\begin{verbatim}
    <<<<<<< .mine
    #ifndef DISTRIBUTE
    IF (Lanafile.and.(tile.eq.0)) THEN
    #else
    IF (Lanafile) THEN
    #endif
    =======
    #ifdef DISTRIBUTE
    IF (Lanafile) THEN
    #else
    IF (Lanafile.and.(tile.eq.0)) THEN
    #endif
    >>>>>>> .r291
\end{verbatim}
After comparing the two code blocks (separated by the ``='' symbols), you
decide that you prefer the logic from the repository so you remove the
conflict markers and your code so the section now looks like this:
\begin{verbatim}
    #ifdef DISTRIBUTE
    IF (Lanafile) THEN
    #else
    IF (Lanafile.and.(tile.eq.0)) THEN
    #endif
\end{verbatim}

Now you can save the file and let Subversion know that you have resolved the
conflict:
\begin{verbatim}
    > svn resolved User/Functionals/ana_hmixcoef.h
    Resolved conflicted state of 'User/Functionals/ana_hmixcoef.h'
\end{verbatim}

\subsubsection{Copying a file onto your working file}
If you get a conflict and decide that you want to throw out your changes,
you can merely copy one of the temporary files created by Subversion over
the file in your working copy. Let's say you want to use the new revision
from the repository:
\begin{verbatim}
    > cd User/Functionals
    > ls ana_hmixcoef.h*
    ana_hmixcoef.h ana_hmixcoef.h.mine ana_hmixcoef.h.r280
    ana_hmixcoef.h.r291

    > cp ana_hmixcoef.h.r291 ana_hmixcoef.h
    > svn resolved ana_hmixcoef.h
    Resolved conflicted state of 'ana_hmixcoef.h'
\end{verbatim}

\subsubsection{Punting: Using svn revert}
If you get a conflict, and upon examination decide that you want to throw
out your changes and start your edits again, just revert your changes:
\begin{verbatim}
    > cd User/Functionals
    > svn revert ana_hmixcoef.h
    Reverted 'ana_hmixcoef.h'
\end{verbatim}
Note that when you revert a conflicted file, you don't have to run
``svn resolved''.
