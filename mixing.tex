\subsection{Horizontal mixing}
\label{Smooth}

In Chapter \ref{Phys}, the diffusive terms were written simply as
${\cal D}_u, {\cal D}_v, {\cal D}_T,$ and ${\cal D}_S$.  The vertical
component of these terms is described in \S\ref{Vfric}.  Here we
describe the ROMS options for representing the horizontal
component of these terms, first the viscosity then the diffusion.

\subsubsection{Deviatory stress tensor (viscosity)}

Note: this material was copied from the wiki, where it was
contributed by Hernan Arango. He uses "$s$" where we have been using
"$\sigma$" while here $\sigma$ is the stress tensor.

The horizontal components of the divergence of the stress tensor
\citep{Wajsowicz_93} in
nondimesional, orthogonal curvilinear coordinates ($\xi$, $\eta$,
$s$) with dimensional, spatially-varying metric factors
($\frac{1}{m}$, $\frac{1}{n}$, $H_{z}$) and velocity components
($u$, $v$, $\omega H_{z}$) are given by:

\begin{align}
      F^{u} \equiv
\widehat{\xi}\cdot\left(\nabla\cdot\vec{\sigma}\right) =
          \frac{mn}{H_{z}} \Biggl[ & {\pder{}{\xi}}  \Biggl(
\frac{H_{z}{\sigma}_{\xi\xi}} {n} \Biggr) +
                                     {\pder{}{\eta}} \Biggl(
\frac{H_{z}{\sigma}_{\xi\eta}}{m} \Biggr) +
                                     {\pder{}{s}}    \Biggl(
\frac{{\sigma}_{\xi s}}{mn} \Biggr) + \\
         &H_{z}{\sigma}_{\xi\eta}  {\pder{}{\eta}} \left(
\frac{1}{m}\right) -
          H_{z}{\sigma}_{\eta\eta} {\pder{}{\xi}}  \left(
\frac{1}{n}\right) -
          \frac{1}{n} {\sigma}_{ss}{\pder{H_{z}}{\xi}} \Biggr]
\label{eqstressu}
\end{align}

\begin{align}
      F^{v} \equiv
\widehat{\eta}\cdot\left(\nabla\cdot\vec{\sigma}\right) =
          \frac{mn}{H_{z}} \Biggl[ & {\pder{}{\xi}}  \Biggl(
\frac{H_{z}{\sigma}_{\eta\xi}} {n} \Biggr) +
                                     {\pder{}{\eta}} \Biggl(
\frac{H_{z}{\sigma}_{\eta\eta}}{m} \Biggr) +
                                     {\pder{}{s}}    \Biggl(
\frac{{\sigma}_{\eta s}}{mn} \Biggr) + \\
         &H_{z}{\sigma}_{\eta\xi}  {\pder{}{\xi}}  \left(
\frac{1}{n} \right) -
          H_{z}{\sigma}_{\xi\xi}   {\pder{}{\eta}} \left(
\frac{1}{m} \right) -
          \frac{1}{m}{\sigma}_{ss} {\pder{H_{z}}{\eta}} \Biggr]
\label{eqstressv}
\end{align}

where
\begin{align}
      {\sigma}_{\xi\xi}   &= \left( A_{M} + \nu \right) e_{\xi\xi} +
\left( \nu - A_{M}\right) e_{\eta\eta}, \\
   \noalign{\smallskip}
      {\sigma}_{\eta\eta} &= \left( \nu - A_{M} \right) e_{\xi\xi} +
\left( A_{M} + \nu\right) e_{\eta\eta}, \\
   \noalign{\smallskip}
      {\sigma}_{ss} &= 2\,\nu\,e_{ss}, \\
   \noalign{\smallskip}
      {\sigma}_{\xi\eta} &= {\sigma}_{\eta\xi} =
2\,A_{M}\,e_{\xi\eta}, \\
   \noalign{\smallskip}
      {\sigma}_{\xi s}   &=  2\,K_{M}\,e_{\xi s}, \\
   \noalign{\smallskip}
      {\sigma}_{\eta s}  &=  2\,K_{M}\,e_{\eta s},
\end{align}

and the strain field is:

\begin{align}
      e_{\xi\xi}   &= m  {\pder{u}{\xi}}  + mnv {\pder{}{\eta}}
\left( \frac{1}{m} \right), \\
   \noalign{\smallskip}
      e_{\eta\eta} &= n\;{\pder{v}{\eta}} + mnu {\pder{}{\xi}}
\left( \frac{1}{n} \right), \\
   \noalign{\smallskip}
      e_{ss} &= \frac{1}{H_{z}}   {\pder{\left( \omega H_{z} \right)
}{s}} + 
                \frac{m}{H_{z}} u {\pder{H_{z}}{\xi}} +
                \frac{n}{H_{z}} v {\pder{H_{z}}{\eta}}, \\
   \noalign{\smallskip}
      2\,e_{\xi\eta} &= \frac{m}{n} {\pder{\left( nv \right) }{\xi}}
+
                        \frac{n}{m} {\pder{\left( mu \right)
}{\eta}}, \\
   \noalign{\smallskip}
      2\,e_{\xi s} &= \frac{1}{mH_{z}}  {\pder{\left( mu \right)
}{s}} +
                 m H_{z} {\pder{\omega}{\xi}}, \\
   \noalign{\smallskip}
      2\,e_{\eta s} &= \frac{1}{nH_{z}} \;
{\pder{\left(nv\right)}{s}} \;+
                 n\;H_{z} {\pder{\omega}{\eta}}.
\end{align}

Here, $A_{M}(\xi,\eta)$ and $K_{M}(\xi,\eta,s)$ are the spatially
varying horizontal and vertical viscosity coefficients,
respectively, and $\nu$ is another (very small, often neglected)
horizontal viscosity coefficient. Notice that because of the
generalized terrain-following vertical coordinates of ROMS, we need
to transform the horizontal partial derivatives from constant
''z-''surfaces to constant ''s-''surfaces.  And the vertical metric
or level thickness is the Jacobian of the transformation,
$H_{z}={\pder{z}{s}}$. Also in these models, the ''vertical''
velocity is computed as $\frac{\omega H_{z}}{mn}$ and has units of
$\hbox{m}^3/\hbox{s}$.

\subsubsection{Transverse stress tensor}

Assuming transverse isotropy, as in
\citet{Sadourny_97} and \citet{Griffies_2000},
the deviatoric stress tensor can be split into vertical and horizontal
sub-tensors.  The horizontal (or transverse) sub-tensor is symmetric, it
has a null trace, and it possesses axial symmetry in the local vertical
direction.  Then, the transverse stress tensor can be derived from eq.
(\ref{eqstressu}) and (\ref{eqstressv}), yielding

\begin{align}
      H_{z}F^{u} &= {n^2}m
{\partial\over\partial\xi}\left(\frac{H_{z}F^{u\xi}}{n}\right) +
                {m^2}n
{\partial\over\partial\eta}\left(\frac{H_{z}F^{u\eta}}{m}\right)
\label{eqtstressu} \\
   \noalign{\smallskip}
      H_{z}F^{v} &= {n^2}m
{\partial\over\partial\xi}\left(\frac{H_{z}F^{v\xi}}{n}\right) +
                {m^2}n
{\partial\over\partial\eta}\left(\frac{H_{z}F^{v\eta}}{m}\right)
\label{eqtstressv} 
\end{align}

where

\begin{align}
         F^{u\xi} &= \frac{1}{n} \;A_{M}\left[
               \frac{m}{n} {\pder{\left( nu \right) }{\xi}} \;-
               \frac{n}{m} {\pder{\left( mv \right) }{\eta}}
\right], \\
      \noalign{\smallskip}
         F^{u\eta} &= \frac{1}{m} A_{M}\left[
               \frac{n}{m} {\pder{\left( mu \right) }{\eta}} +
               \frac{m}{n} {\pder{\left( nv \right) }{\xi}}
\;\right], \\
      \noalign{\medskip}
         F^{v\xi} &= \frac{1}{n} \;A_{M}\left[
               \frac{m}{n} {\pder{\left( nv \right) }{\xi}} \;+
               \frac{n}{m} {\pder{\left( mu \right) }{\eta}}
\right], \\
      \noalign{\smallskip}
         F^{v\eta} &= \frac{1}{m} A_{M}\left[
               \frac{n}{m} {\pder{\left( mv \right) }{\eta}} -
               \frac{m}{n} {\pder{\left( nu \right) }{\xi}}
\;\right].
\end{align}

Notice the flux form of eq. (\ref{eqtstressu}) and (\ref{eqtstressv})
and the symmetry between the $F^{u\xi}$ and $F^{v\eta}$ terms which are
defined at density points on a C-grid. Similarly, the $F^{u\eta}$ and
$F^{v\xi}$ terms are symmetric and defined at vorticity points.  These
staggering positions are optimal for the discretization of the tensor;
it has no computational modes and satisfies first-moment conservation.

The biharmonic friction operator can be computed by applying the tensor
operator eq. (\ref{eqtstressu}) and (\ref{eqtstressv}) twice, but with
the squared root of the biharmonic viscosity coefficient
\citep{Griffies_2000}.  For simplicity and momentum
balance, the thickness $H_{z}$ appears only when computing the second
harmonic operator as in \citet{Griffies_2000}.

\subsubsection{Rotated Transverse Stress Tensor}

In some applications with tall and steep topography, it
will be advantageous to substantially reduce the contribution of the
stress tensor eq. (\ref{eqtstressu}) and (\ref{eqtstressv}) to
the vertical mixing when operating along constant $s$-surfaces.
The transverse stress tensor rotated along geopotentials (constant depth)
is then given by

\begin{align}
      H_{z}R^{u} &= {n^2}m {\pder{}{\xi}}  \Biggl(
\frac{H_{z}R^{u\xi}} {n} \Biggr) +
                    {m^2}n {\pder{}{\eta}} \Biggl(
\frac{H_{z}R^{u\eta}}{m} \Biggr) +
                           {\pder{}{s}}    \Biggl( R^{us} \Biggr)
\label{eqrstressu}
\\
   \noalign{\smallskip}
      H_{z}R^{v} &= {n^2}m {\pder{}{\xi}}  \Biggl(
\frac{H_{z}R^{v\xi}} {n} \Biggr) +
                    {m^2}n {\pder{}{\eta}} \Biggl(
\frac{H_{z}R^{v\eta}}{m} \Biggr) +
                           {\pder{}{s}}    \Biggl( R^{vs} \Biggr)
\label{eqrstressv}
\end{align}

where

\begin{align}
         R^{u\xi} = &\frac{1}{n}\; A_{M} \left[
                 \frac{1}{n}\;\left( m {\pder{\left(nu\right)}{\xi}} -
                                     m {\pder{z}{\xi}} \frac{1}{H_{z}}
                         {\pder{\left( nu \right) }{s}} \right) -
                 \frac{1}{m}  \left( n {\pder{\left( mv \right) }{\eta}} -
                                     n {\pder{z}{\eta}} \frac{1}{H_{z}}
                                       {\pder{\left( mv \right) }{s}}\right)
                              \right], \\
      \noalign{\medskip}
         R^{u\eta} = &\frac{1}{m} A_{M} \left[
                 \frac{1}{m}  \left( n {\pder{\left( mu \right) }{\eta}} -
                                     n {\pder{z}{\eta}} \frac{1}{H_{z}}
                         {\pder{\left( mu \right) }{s}} \right) +
                 \frac{1}{n}\;\left( m {\pder{\left( nv \right) }{\xi}} -
                                     m {\pder{z}{\xi}} \frac{1}{H_{z}}
                                      {\pder{\left( nv \right) }{s}} \right)
                              \right], \\
      \noalign{\medskip}
         R^{us} = &m {\pder{z}{\xi}} A_{M} \left[
                 \frac{1}{n}\;\left( m {\pder{z}{\xi}} \frac{1}{H_{z}}
                                       {\pder{\left( nu \right) }{s}} -
                 m {\pder{\left( nu \right) }{\xi}} \right) -
                 \frac{1}{m}  \left( n {\pder{z}{\eta}} \frac{1}{H_{z}}
                                       {\pder{\left( mv \right) }{s}} -
                           n {\pder{\left( mv \right) }{\eta}} \right)
                              \right] +\\
                &n\; {\pder{z}{\eta}} A_{M} \left[
                 \frac{1}{m}  \left( n {\pder{z}{\eta}} \frac{1}{H_{z}}
                                       {\pder{\left( mu \right) }{s}} -
                             n {\pder{\left( mu \right) }{\eta}} \right) +
                 \frac{1}{n}\;\left( m {\pder{z}{\xi}} \frac{1}{H_{z}}
                                       {\pder{\left( nv \right) }{s}} -
                             m {\pder{\left( nv \right) }{\xi}} \right)
                              \right], \\
      \noalign{\bigskip}
         R^{v\xi} = &\frac{1}{n}\;A_{M} \left[
                 \frac{1}{n}\;\left( m {\pder{\left( nv \right) }{\xi}} -
                                     m {\pder{z}{\xi}} \frac{1}{H_{z}}
                            {\pder{\left( nv \right) }{s}} \right) +
                 \frac{1}{m}  \left( n {\pder{\left( mu \right) }{\eta}}-
                                     n {\pder{z}{\eta}} \frac{1}{H_{z}}
                                      {\pder{\left( mu \right) }{s}} \right)
                              \right], \\
      \noalign{\medskip}
         R^{v\eta} = &\frac{1}{m}A_{M} \left[
                 \frac{1}{m}  \left( n {\pder{\left( mv \right) }{\eta}} -
                                     n {\pder{z}{\eta}} \frac{1}{H_{z}}
                              {\pder{\left( mv \right) }{s}} \right) -
                 \frac{1}{n}\;\left( m {\pder{\left( nu \right) }{\xi}} -
                                     m {\pder{z}{\xi}} \frac{1}{H_{z}}
                                       {\pder{\left( nu \right)}{s}} \right)
                              \right], \\
      \noalign{\medskip}
         R^{vs} = &m {\pder{z}{\xi}} A_{M} \left[
                 \frac{1}{n}\;\left( m {\pder{z}{\xi}} \frac{1}{H_{z}}
                                       {\pder{\left( nv \right) }{s}} -
                        m {\pder{\left( nv \right) }{\xi}} \right) +
                 \frac{1}{m}  \left( n {\pder{z}{\eta}} \frac{1}{H_{z}}
                                       {\pder{\left( mu \right) }{s}} -
                            n {\pder{\left( mu \right) }{\eta}} \right)
                              \right] +\\
                &n\; {\pder{z}{\eta}} A_{M} \left[
                 \frac{1}{m}  \left( n {\pder{z}{\eta}} \frac{1}{H_{z}}
                                       {\pder{\left( mv \right) }{s}} -
                              n {\pder{\left( mv \right) }{\eta}} \right) -
                 \frac{1}{n}\;\left( m {\pder{z}{\xi}} \frac{1}{H_{z}}
                                       {\pder{\left( nu \right) }{s}} -
                            m {\pder{\left( nu \right) }{\xi}} \right)
                              \right].
\end{align}

Notice that the transverse stress tensor remains invariant under
coordinate transformation.  The rotated tensor (\ref{eqrstressu})
and (\ref{eqrstressv}) retains the
same properties as the unrotated tensor (\ref{eqtstressu}) and
(\ref{eqtstressv}).  The additional terms
that arise from the slopes of $s$-surfaces along
geopotentials are discretized using a modified version of the triad
approach of \citet{Griffies_98}.

\subsubsection{Horizontal diffusion}
\label{Smooth_diff}

In Chapter \ref{Phys}, the diffusive terms were written simply as
${\cal D}_T$ and ${\cal D}_S$. The vertical component of these terms
is described in \S\ref{Vfric}. Here we describe the various options
for representing the horizontal component of these terms.

\subsubsection{Laplacian}
The Laplacian of a scalar $C$ in curvilinear coordinates is (see
Appendix \ref{Curve}):
\begin{equation}
   \nabla^2 C = \nabla \cdot \nabla C = mn \left[ 
   {\partial \over \partial \xi} \!\! \left( {m \over n} 
   {\partial C \over \partial \xi} \right) +
   {\partial \over \partial \eta} \!\! \left( {n \over m} 
   {\partial C \over \partial \eta} \right) \right]
\end{equation}
In ROMS, this term is multiplied by ${\nu_2 H_z \over mn}$ and becomes
\begin{equation}
   \left[ 
   {\partial \over \partial \xi} \!\! \left( {\nu_2 H_zm \over n} 
   {\partial C \over \partial \xi} \right) +
   {\partial \over \partial \eta} \!\! \left( {\nu_2 H_zn \over m} 
   {\partial C \over \partial \eta} \right) \right]
\end{equation}
where $C$ is any tracer. This form guarantees
that the term does not contribute to the volume-integrated equations.

\subsubsection{Biharmonic}
The biharmonic operator is $\nabla^4 = \nabla^2 \nabla^2$; the
corresponding term is computed using a temporary variable $Y$:
\begin{equation}
   Y =
   {mn \over H_z} \left[ 
   {\partial \over \partial \xi} \!\! \left( {\nu_4 H_zm \over n} 
   {\partial C \over \partial \xi} \right) +
   {\partial \over \partial \eta} \!\! \left( {\nu_4 H_zn \over m} 
   {\partial C \over \partial \eta} \right) \right]
\end{equation}
and is
\begin{equation}
   - \left[ 
   {\partial \over \partial \xi} \!\! \left( {\nu_4 H_zm \over n} 
   {\partial Y \over \partial \xi} \right) +
   {\partial \over \partial \eta} \!\! \left( {\nu_4 H_zn \over m} 
   {\partial Y \over \partial \eta} \right) \right]
\end{equation}
where $C$ is once again any tracer and $\nu_4$ is the square root of
the input value so that it can be applied twice.

\subsubsection{Rotated mixing tensors}
Both the Laplacian and biharmonic terms above operate on surfaces of
constant $s$ and can contribute substantially to the vertical mixing.
However, the oceans are thought to mix along constant density surfaces so
this is not entirely satisfactory. Therefore, the option of using rotated
mixing tensors for the Laplacian and biharmonic operators has been added.
Options exist to diffuse on constant $z$ surfaces (\code{MIX\_GEO\_TS})
and constant potential density surfaces (\code{MIX\_ISO\_TS}).

The horizontal Laplacian diffusion operator is computed by finding the
three components of the flux of the quantity $C$.  The $\xi$ and
$\eta$ components are locally horizontal, rather than along the
$s$ surface. The diffusive fluxes are:
\begin{align}
   F^\xi & = \nu_2 \left[ m
   {\partial C \over \partial \xi} -
   \underbrace{ \left( m {\partial z \over \partial \xi}
   \underbrace{+ S_x }_{\mbox{\tt MIX\_ISO}} \right)
   {\partial C \over \partial z} }_{\mbox{\tt MIX\_GEO}} \right]
\label{fluxx}
\\
   F^\eta & = \nu_2 \left[ n
   {\partial C \over \partial \eta} -
   \underbrace{ \left[ n {\partial z \over \partial \eta}
   \underbrace{+ S_y }_{\mbox{\tt MIX\_ISO}} \right)
   {\partial C \over \partial z} }_{\mbox{\tt MIX\_GEO}} \right]
\label{fluxy}
\\
   F^s & =
   - \underbrace{ {1 \over H_z} \left( m
   {\partial z \over \partial \xi}
   \underbrace{+ S_x }_{\mbox{\tt MIX\_ISO}} \right)
   F^\xi }_{\mbox{\tt MIX\_GEO}}
   - \underbrace{ {1 \over H_z} \left( n
   {\partial z \over \partial \eta}
   \underbrace{+ S_y }_{\mbox{\tt MIX\_ISO}} \right)
   F^\eta }_{\mbox{\tt MIX\_GEO}}
\label{fluxs}
\end{align}
where
\begin{align*}
  S_x & = { { \partial \rho \over \partial x} \over
    { \partial \rho \over \partial z} } =
    { \left[ m {\partial \rho \over \partial \xi} -
    {m \over H_z} {\partial z \over \partial \xi}
    { \partial \rho \over \partial s} \right] \over
    {1 \over H_z} {\partial \rho \over \partial s} }
\\
  S_y & = { { \partial \rho \over \partial y} \over
    { \partial \rho \over \partial z} } =
    { \left[ n {\partial \rho \over \partial \eta} -
    {n \over H_z} {\partial z \over \partial \eta}
    { \partial \rho \over \partial s} \right] \over
    {1 \over H_z} {\partial \rho \over \partial s} }
\end{align*}
and there is some trickery whereby the computational details depend on the sign
of $\partial z \over \partial \xi$ and of $\partial z \over \partial
\eta$.  No flux boundary conditions are easily imposed by setting
\begin{eqnarray*}
  F^\xi = 0    && \mbox{ at $\xi$ walls} \\
  F^\eta = 0   && \mbox{ at $\eta$ walls} \\
  F^s = 0 && \mbox{ at $s = -1,0$}
\end{eqnarray*}

Finally, the flux divergence is calculated and is added to the
right-hand-side term for the field being computed:
\begin{equation}
  {\partial \over \partial \xi} \left( { H_z F^\xi \over n} \right) +
  {\partial \over \partial \eta} \left( { H_z F^\eta \over m} \right) +
  {\partial \over \partial s}
  \left( { H_z F^s \over mn} \right)
\label{rot1}
\end{equation}

The biharmonic rotated mixing tensors are computed much as the
non-rotated biharmonic mixing.  We define a temporary variable $Y$
based on equation (\ref{rot1}):
\begin{equation}
  Y = {mn \over H_z} \left[
  {\partial \over \partial \xi} \left( {\nu_4 H_z F^\xi \over n} \right) +
  {\partial \over \partial \eta} \left( {\nu_4 H_z F^\eta \over m} \right) +
  {\partial \over \partial s}
  \left( {\nu_4 H_z F^s \over mn} \right)
  \right] .
\end{equation}
We then build up fluxes of $Y$ as in equations
(\ref{fluxx})--(\ref{fluxs}).  We then apply equation (\ref{rot1})
to these $Y$ fluxes to obtain the biharmonic mixing tensors. Again, the
value of $\nu_4$ is the square root of that read in so that it can be
applied twice.

\subsection{Vertical mixing schemes}
\label{Vmix}
ROMS contains a variety of methods for setting the vertical viscous and
diffusive coefficients. The choices range from simply choosing fixed
values to the KPP, generic lengthscale (GLS) and Mellor-Yamada turbulence
closure schemes.  See \citet{Large98} for a review of surface ocean
mixing schemes.  Many schemes have a background molecular value which
is used when the turbulent processes are assumed to be small (such as
in the interior). All assume that there is some $K_m(\zeta,\eta,s)$
such that the vertical turbulent mixing can be applied as:
\begin{equation}
    \overline{u' w'} = -K_m {partial u \over \partial z}
    \mbox{\quad and \quad}
    \overline{v' w'} = -K_m {partial v \over \partial z}
\end{equation}
with a similar $K_s$ for temperature, salinity and other tracers. The
primed quantities represent perturbations of smaller scale than those
resolved by the model while the unprimed $u$ and $v$ are those in
the model.

\subsubsection{The Large, McWilliams and Doney parameterization}
\label{sec:origLMD}

The vertical mixing parameterization introduced by
\citet{Large94} is a versatile first order scheme which has
been shown to perform well in open ocean settings.  Its design
facilitates experimentation with additional or modified representations
of specific turbulent processes.

\paragraph{Surface boundary layer}
The Large, McWilliams and Doney scheme (LMD)
matches separate parameterizations for vertical mixing
of the surface boundary layer and the ocean interior.  A formulation
based on boundary layer similarity theory is applied in the water
column above a calculated boundary layer depth $h_{sbl}$.  This
parameterization is then matched at the interior with schemes to
account for local shear, internal wave and double diffusive mixing
effects.  

Viscosity and diffusivities at model levels above a calculated
surface boundary layer depth ($h_{sbl}$ ) are expressed as the product
of the length scale $h_{sbl}$, a turbulent velocity scale $w_x$ and a
non-dimensional shape function.
\begin{equation}
\nu_x = h_{sbl} w_x(\sigma)G_x(\sigma)
\end{equation}
where $\sigma$ is a non-dimensional coordinate ranging from 0 to 1
indicating depth within the surface boundary layer. The $x$ subscript
stands for one of momentum, temperature and salinity.

\subparagraph{Surface Boundary layer depth}
The boundary layer depth $h_{sbl}$ is calculated as the minimum of the
Ekman depth, estimated as,
\begin{equation}
h_e=0.7u_*/f
\end{equation}
(where $u_*$ is the friction velocity $u_*=\sqrt{\tau_x^2+\tau_y^2}/\rho$ ),
 the Monin-Obukhov depth:
\begin{equation}
L=u_*^3/(\kappa B_f)
\end{equation}
(where $\kappa = 0.4$ is von Karman's contant and $B_f$ is the surface
buoyancy flux), and the shallowest depth at which a critical bulk
Richardson number is reached. The critical bulk Richardson number
($Ri_c$) is typically in the range 0.25--0.5. The bulk Richardson
number ($Ri_b$) is calculated as:
\begin{equation}
Ri_b(z)=\frac{(B_r-B(d))d}{|\vec{V}_r-\vec{V}(d)|^2+{V_t}^2(d)}
\end{equation}
where $d$ is distance down from the surface, $B$ is the buoyancy,
$B_r$ is the buoyancy at a near surface reference depth, $\vec{V}$ is
the mean horizontal velocity, $\vec{V}_r$ the velocity at the near
surface reference depth and $V_t$ is an estimate of the turbulent
velocity contribution to velocity shear.

The turbulent velocity shear term in this equation is given by LMD as,
\begin{equation}
  V_{t}^{2}(d)=\frac{C_v(-\beta_T)^{1/2}}{Ri_c
  \kappa}(c_s\epsilon)^{-1/2}dNw_s
  \label{eqtvs}
\end{equation}
where $C_v$ is the ratio of interior $N$ to $N$ at the entrainment
depth, $\beta_T$ is ratio of entrainment flux to surface buoyancy flux,
$c_s$ and $\epsilon$ are constants, and $w_s$ is the turbulent velocity
scale for scalars.
LMD derive (\ref{eqtvs}) based on the expected behavior in the pure
convective limit.  The empirical rule of convection states that the
ratio of the surface buoyancy flux to that at the entrainment depth be 
a constant.  Thus the entrainment flux at the
bottom of the boundary layer under such conditions should be
independent of the stratification at that depth.  Without a turbulent
shear term in the denominator of the bulk Richardson number
calculation, the estimated boundary layer depth is too shallow and the
diffusivity at the entrainment depth is too low to obtain the
necessary entrainment flux.  Thus by adding a turbulent shear term
proportional to the stratification in the denominator, the calculated
boundary layer depth will be deeper and will lead to a high enough
diffusivity to satisfy the empirical rule of convection.
  
\subparagraph{Turbulent velocity scale}
To estimate $w_x$ (where $x$ is $m$ - momentum {\em or} $s$
- any scalar) throughout the boundary layer, surface layer similarity
theory is utilized.  Following an argument by
\citet{TM86}, \citet{Large94} estimate the velocity scale as
\begin{equation}
w_x=\frac{\kappa u_*}{\phi_x(\zeta)}
\end{equation}
where $\zeta$ is the surface layer stability parameter defined as
$z/L$.  $\phi_x$ is a non-dimensional flux profile which varies based
on the stability of the boundary layer forcing.  The stability
parameter used in this equation is assumed to vary over the entire
depth of the boundary layer in stable and neutral conditions.  In
unstable conditions it is assumed only to vary through the surface
layer which is defined as $ \epsilon h_{sbl} $ (where $\epsilon$ is
set at 0.10) .  Beyond this depth $\zeta$
is set equal to its value at $ \epsilon h_{sbl} $.

The flux profiles are expressed as analytical fits to atmospheric
surface boundary layer data.  In stable conditions they vary linearly
with the stability parameter $\zeta$  as
\begin{equation}
\phi_x=1+5\zeta
\end{equation}
In near-neutral unstable conditions common Businger-Dyer forms are
used which match with the formulation for stable conditions at
$\zeta=0$.  Near neutral conditions are defined as 
\begin{equation}
-0.2 \leq \zeta < 0
\end{equation}
for momentum and,
\begin{equation}
-1.0 \leq \zeta < 0
\end{equation}
for scalars.  The non dimensional flux profiles in this regime are,
\begin{align}
\phi_m & =(1-16\zeta)^{1/4}
\\ \vspace{1mm}
\phi_s & =(1-16\zeta)^{1/2}
\end{align}
In more unstable conditions $\phi_x$ is chosen to match the
Businger-Dyer forms and with the free convective limit.  Here the flux 
profiles are 
\begin{align}
\phi_m & =(1.26-8.38\zeta)^{1/3}
\\ \vspace{1mm}
\phi_s & =(-28.86-98.96\zeta)^{1/3}
\end{align}

\subparagraph{The shape function}
The non-dimensional shape function $G(\sigma)$ is a third order
polynomial with coefficients chosen to match the interior viscosity at
the bottom of the boundary layer and Monin-Obukhov
similarity theory approaching the surface.  This function is defined
as a 3rd order polynomial.
\begin{equation}
G(\sigma)=a_o+a_1\sigma+a_2\sigma^2+a_3\sigma^3
\end{equation}
with the coefficients specified to match surface boundary conditions
and to smoothly blend with the interior,
\begin{align}
  a_o & =0
\\ \vspace{1mm}
  a_1 & =1
\\ \vspace{1mm}
  a_2 & =-2+3\frac{\nu_{x}(h_{sbl})}{hw_x(1)}+\frac{\partial_x
  \nu_{x}(h)}{w_{x}(1)}+\frac{\nu_{x}(h)
  \partial_{\sigma}w_x(1)}{hw_{x}^{2}(1)}
\\ \vspace{1mm}
  a_3 & =1-2\frac{\nu_{x}(h_{sbl})}{hw_x(1)}-\frac{\partial_x
  \nu_{x}(h)}{w_{x}(1)}-\frac{\nu_{x}(h)
  \partial_{\sigma}w_x(1)}{hw_{x}^{2}(1)}
\end{align}
where $\nu_{x}(h)$ is the viscosity calculated by the interior
parameterization at the boundary layer depth.

\paragraph{Countergradient flux term}
The second term of the LMD scheme's surface boundary layer
formulation is the non-local transport term $\gamma$ which can play a
significant role in mixing during surface cooling events.  This is a
redistribution term included in the tracer equation separate from the
diffusion term and is written as 
\begin{equation}
-\frac{\partial}{\partial z}K\gamma.
\end{equation}

LMD base their formulation for non-local scalar transport on a
parameterization for pure free convection from
\citet{Mailhot82}. They extend this parameterization to cover any
unstable surface forcing conditions to give
\begin{equation}
  \gamma_{T}=C_s\frac{\overline{wT_0}+
  \overline{wT_R}}{w_T(\sigma)h}
\end{equation}
for temperature and 
\begin{equation}
\gamma_S=C_s \frac{\overline{wS_0}}{w_S(\sigma)h}
\end{equation}
for salinity (other scalar quantities with surface fluxes can be
treated similarly). LMD argue that although there is evidence of
non-local transport of momentum as well, the form the term would take
is unclear so they simply specify $\gamma_m=0$.

\paragraph{The interior scheme}
The interior scheme of Large, McWilliams and Doney estimates the
viscosity coefficient by adding the effects of several generating
mechanisms:  shear mixing, double-diffusive mixing and internal wave
generated mixing.
\begin{equation}
\nu_{x}(d)=\nu_{x}^s+\nu_{x}^d+\nu_{x}^w
\end{equation}

\subparagraph{Shear generated mixing}
The shear mixing term is calculated using a
gradient Richardson number formulation,
with viscosity estimated as: 
\begin{equation}
\nu^s_x=\begin{cases}
\nu_0&   \text{$ Ri_g<0$}, \\
\nu_0[1-(Ri_g/Ri_0)^2]^3&  \text{$0< Ri_g<Ri_0$},  \\
0&   \text{$Ri_g>Ri_0$}.  
\end{cases}
\end{equation}
where $\nu_0$ is $5.0 \times 10^{-3}$, $Ri_0 = 0.7$.  

\subparagraph{Double diffusive processes}
The second component of the interior mixing parameterization represents
double diffusive mixing.  From limited sources of laboratory and field
data LMD parameterize the salt fingering case ($R_{\rho}>1.0$)
\begin{align}
\nu_{s}^{d}(R_{\rho}) & =
	\begin{cases}
      1\times10^{-4}[1-(\frac{(R_{\rho}-1}{R_{\rho}^0-1})^2)^{3}&
      \text{for $1.0<R_{\rho}<R_{\rho}^0=1.9$},\\
           0& \text{otherwise}.
        \end{cases}
\\ \vspace{1mm}
\nu_{\theta}^{d}(R_{\rho}) & =0.7\nu_{s}^{d}
\end{align}

For diffusive convection ($0<R_{\rho}<1.0$) LMD suggest several
formulations from the literature and choose the one with the most
significant impact on mixing \citep{Fedorov88}.
\begin{equation}
\nu_{\theta}^{d}=(1.5 \time 10^{-6})(0.909 \exp(4.6 \exp[-0.54(R_{\rho}^{-1}-1)])
\end{equation}
for temperature.  For other scalars,
\begin{equation}
   \nu_{s}^{d}=
	\begin{cases}
	     \nu_{\theta}^{d}(1.85-0.85R_{\rho}^{-1})R_{\rho}& \text{for $0.5<=R_{\rho}<1.0$},\\ 
             \nu_{\theta}^{d}0.15R_{\rho}&  \text{otherwise}. \\
        \end{cases}
\end{equation}

\subparagraph{Internal wave generated mixing}
Internal wave generated mixing serves as the background mixing in the
LMD scheme.  It is specified as a constant for both scalars and
momentum.  Eddy diffusivity is estimated based on the data of
\citet{LWL93}.  While \citet{Peters88} suggest
eddy viscosity should be 7 to 10 times larger than diffusivity for
gradient Richardson numbers below approximately 0.7.  Therefore LMD use
\begin{align}
\nu_{m}^w & =1.0 \times 10^{-4} m^2 s^{-1}
\\ \vspace{1mm}
\nu_{s}^w & =1.0 \times 10^{-5} m^2 s^{-1}
\end{align}

\subsubsection{Mellor-Yamada}
\label{sec:MY25}
One of the more popular closure schemes is that of
\citet{Mellor74, Mellor82}. They actually present a hierarchy of
closures of increasing complexity. ROMS provides only the
``Level 2.5'' closure with the \citet{Galperin88}
modifications as described in \citet{Allen95}.
This closure scheme adds two prognostic equations, one
for the turbulent kinetic energy (${1 \over 2} q^2$) and one for the
turbulent kinetic energy times a length scale ($q^2l$).

The turbulent kinetic energy equation is:
\begin{equation}
  {D \over Dt} \left( {q^2 \over 2} \right) -
  {\partial \over \partial z} \left[ K_q {\partial \over \partial z} 
  \left( {q^2 \over 2} \right) \right] = P_s + P_b - \xi_d
  \label{eq:tke1}
\end{equation}
where $P_s$ is the shear production, $P_b$ is the buoyant production
and $\xi_d$ is the dissipation of turbulent kinetic energy. 
These terms are given by
\begin{align}
   P_s &= K_m \left[ \left( {\partial u \over \partial z }\right)^2 +
   \left( {\partial v \over \partial z} \right)^2 \right],  \\
   P_b &= -K_s N^2, \\
   \xi_d &= {q^3 \over B_1 l}
\end{align}
where $B_1$ is a constant.
One can also add a traditional horizontal Laplacian or biharmonic
diffusion (${\cal D}_q$) to the turbulent kinetic energy equation.
The form of this equation in the model coordinates becomes
%{\samepage
\begin{multline}
  {\partial \over \partial t} \left( {H_z q^2 \over mn} \right) +
  {\partial \over \partial \xi} \left( {H_z u q^2 \over n} \right) +
  {\partial \over \partial \eta} \left( {H_z v q^2 \over m} \right) +
  {\partial \over \partial s} \left( {H_z \Omega q^2 \over mn} \right) -
  {\partial \over \partial s} \left( {K_q \over mnH_z}
  {\partial q^2 \over \partial s} \right) =
\\ \vspace{1mm}
  {2H_z K_m \over mn} \left[ \left({\partial u \over \partial z}
  \right)^2 + \left( {\partial v \over \partial z} \right)^2 \right] +
  {2H_z K_s \over mn} N^2 - {2H_z q^3 \over mnB_1 l} +
  {H_z \over mn} {\cal D}_q .
  \label{eq:tke2}
\end{multline}
%}
The vertical boundary conditions are:
\[
\begin{array}{rl}
  \mbox{top ($z = \zeta(x,y,t))$} \hspace{1cm}
  & {H_z \Omega \over mn} = 0 \\ [1.5mm]
  & {K_q \over mn H_z} \, \frac{\partial q^2}{\partial s}
    = {B_1^{2/3} \over \rho_o} \left[ \left( \tau_s^\xi \right)^2
    + \left( \tau_s^\eta \right)^2  \right] \\ [1.5mm]
  & H_z K_m \left( {\partial u \over \partial z},
    {\partial v \over \partial z} \right) = {1 \over \rho_o}
    \left( \tau_s^\xi, \tau_s^\eta \right) \\ [1.5mm]
  & H_z K_s N^2 = {Q \over \rho_o c_P} \\[2mm]
  \mbox{and bottom ($z = -h(x,y)$)} \hspace{1cm} &
    {H_z \Omega \over mn} = 0 \\[1.5mm]
  & {K_q \over mnH_z} {\partial q^2 \over \partial s}
    = {B_1^{2/3} \over \rho_o} \left[ \left( \tau_b^\xi \right)^2
    + \left( \tau_b^\eta \right)^2  \right] \\ [1.5mm]
  & H_z K_m \left( {\partial u \over \partial z},
    {\partial v \over \partial z} \right) = {1 \over \rho_o}
    \left( \tau_b^\xi, \tau_b^\eta \right) \\ [1.5mm]
  & H_z K_s N^2 = 0
\end{array}
\]

There is also an equation for the turbulent length scale $l$:
\begin{equation}
  {D \over Dt} \left( {lq^2} \right) -
  {\partial \over \partial z} \left[ K_l
  {\partial lq^2 \over \partial z} 
  \right] = lE_1 ( P_s + P_b ) - {q^3 \over B_1} \tilde{W}
\end{equation}
where $\tilde{W}$ is the wall proximity function:
\begin{align}
  \tilde{W} &= 1 + E_2 \left( {l \over kL} \right) ^2 \\
  L^{-1} &= {1 \over \zeta -z} + {1 \over H+z}
\end{align}
The form of this equation in the model coordinates becomes
%{\samepage
\begin{multline}
  {\partial \over \partial t} \left( {H_z q^2l \over mn} \right) +
  {\partial \over \partial \xi} \left( {H_z u q^2l \over n} \right) +
  {\partial \over \partial \eta} \left( {H_z v q^2l \over m} \right) +
  {\partial \over \partial s} \left( {H_z \Omega q^2l \over mn} \right) -
  {\partial \over \partial s} \left( {K_q \over mnH_z}
  {\partial q^2l \over \partial s} \right) =
\\ \vspace{1mm}
  {H_z \over mn} lE_1 ( P_s + P_b) - 
  {H_z q^3 \over mnB_1 } \tilde{W} +
  {H_z \over mn} {\cal D}_{ql} .
  \label{eq:kkl}
\end{multline}
%}
where ${\cal D}_{ql}$ is the horizontal diffusion of the quantity
$q^2l$. Both equations (\ref{eq:tke2}) and (\ref{eq:kkl}) are timestepped
much like the model tracer equations, including an implicit solve for
the vertical operations and options for centered second and fourth-order
advection. They are timestepped with a predictor-corrector scheme in
which the predictor step is only computing the advection.

Given these solutions for $q$ and $l$, the vertical viscosity and
diffusivity coefficients are:
\begin{align}
  K_m &= qlS_m + K_{m_{\rm background}} \\
  K_s &= qlS_h + K_{s_{\rm background}} \\
  K_q &= qlS_q + K_{q_{\rm background}}
\end{align}
and the stability coefficients $S_m$, $S_h$ and $S_q$ are found by
solving
\begin{gather}
  S_s \left[ 1 - (3A_2 B_2 + 18 A_1 A_2) G_h \right] =
  A_2 \left[ 1 - 6A_1 B_1^{-1} \right]
\\ \vspace{1mm}
  S_m \left[ 1 - 9A_1 A_2 G_h \right] - S_s \left[ G_h ( 18 A_1^2 +
  9A_1 A_2 ) G_h \right] =
  A_1 \left[ 1 - 3C_1 - 6A_1 B_1^{-1} \right]
\\ \vspace{1mm}
  G_h = \min ( -{l^2N^2 \over q^2}, 0.028 ).
\\ \vspace{1mm}
  S_q = 0.41 S_m
\end{gather}
The constants are set to $(A_1, A_2, B_1, B_2, C_1, E_1, E_2) = 
(0.92, 0.74, 16.6, 10.1, 0.08, 1.8, 1.33)$. The quantities $q^2$ and
$q^2l$ are both constrained to be no smaller than $10^{-8}$ while $l$
is set to be no larger than $0.53q/N$.

\subsubsection{Generic length scale}
\citep{Umlauf2003} have come up with a generic
two-equation turbulence closure scheme which can be tuned to behave
like several of the traditional schemes, including that of Mellor
and Yamada \S(\ref{sec:MY25}). This is known as the Generic Length
Scale, or GLS vertical mixing scheme and was introduced to ROMS in
\citep{Warner_2005}. Its parameters are set in the
ROMS input file.

The first of Warner et al.'s equations is the same as (\ref{eq:tke1})
with $k=1/2 q^2$. Their dissipation is given by
\begin{equation}
  \epsilon = (c^0_\mu ) ^{3+p/n} k^{3/2+m/n} \psi ^{-1/n}
\end{equation}
where $\psi$ is a generic parameter that is used to establish the
turbulence length scale. The equation for $\psi$ is:
\begin{equation}
  {D \psi \over D t} = {\partial \over \partial z} \left( K_\psi  
  {\partial \psi \over \partial z}\right) + {\psi \over k}
  (c_1 P_s + c_3 P_b - c_2 \epsilon F_{\rm wall})
\end{equation}
Coefficients $c_1$ and $c_2$ are chosen to be consistent
with observations of decaying homogeneous, isotropic turbulence. The
parameter $c_3$ has differing values for stable ($c^+_3$) and unstable
($c^-_3$) stratification. Also
\begin{eqnarray}
   \psi &= (c^0_\mu)^p k^m l^n \\
   l &= (c^0_\mu)^3 k^{3/2} \epsilon{-1}
\end{eqnarray}

Depending on the choice of the various parameters, these two equations can
be made to solve a variety of traditional two-equation turbulence closure
models. The list of parameters is shown in table \ref{t:gls} and is also
given inside the comments section of the ROMS input file.

\begin{table}[thb]
\centerline{
\begin{tabular}{|l|l|l|l|l|} \hline
& $k-kl$ & $k-\epsilon$ & $k-\omega$ & gen \\
& $\psi = k^1 l^1$ & $\psi = (c^0_\mu)^3 k^{3/2} l^{-1}$ &
$\psi = (c^0_\mu)^{-1} k^{1/2} l^{-1}$ & $\psi = (c^0_\mu)^2 k^1 l^{-2/3}$ \\
  \hline
  p & 0.0 & 3.0 & -1.0 & 2.0 \\
  m & 1.0 & 1.5 & 0.5 & 1.0 \\
  n & 1.0 & -1.0 & -1.0 & -0.67 \\
  $\sigma_k = {K_M \over K_k}$ & 1.96 & 1.0 & 2.0 & 0.8 \\
  $\sigma_\psi = {K_M \over K_\psi}$ & 1.96 & 1.3 & 2.0 & 1.07 \\
  $c_1$ & 0.9 & 1.44 & 0.555 & 1.0 \\
  $c_2$ & 0.52 & 1.92 & 0.833 & 1.22 \\
  $c^-_3$ & 2.5 & -0.4 & -0.6 & 0.1 \\
  $c^+_3$ & 1.0 & 1.0 & 1.0 & 1.0 \\
  $k_{min}$ & 5.0e-6 & 7.6e-6 & 7.6e-6 & 1.0e-8 \\
  $\psi_{min}$ & 5.0e-6 & 1.0e-12 & 1.0e-12 & 1.0e-8 \\
  $c^0_\mu$ & 0.5544 & 0.5477 & 0.5477 & 0.5544 \\
  \hline
\end{tabular}
}
\label{t:gls}
\caption{Generic length scale parameters.
Note that Mellor-Yamada 2.5 is an example of a $k-kl$ scheme.}
\end{table}

\subsection{Timestepping vertical viscosity and diffusion} \label{Vfric}
The ${\cal D}_u$, ${\cal D}_v$, and ${\cal D}_C$ terms in equations
(\ref{st13})--(\ref{st15}) represent both horizontal and vertical mixing
processes.  The horizontal options were covered in \S\ref{Smooth}. The
model has several options for computing the vertical coefficients;
these will be described in \S\ref{Vmix}.  The vertical viscosity and
diffusion terms have the form: \begin{equation}
   {\partial \over \partial \sigma} \left( {K \over H_z mn} {\partial
   \phi \over \partial \sigma} \right)
\end{equation} where $\phi$ represents one of $u$, $v$, or $C$, and $K$
is the corresponding vertical viscous or diffusive coefficient. This is
timestepped using a semi-implicit Crank-Nicholson scheme with a weighting
of 0.5 on the old timestep and 0.5 on the new timestep.  Specifically,
the equation of motion for $\phi$ can be written as: \begin{equation}
  {\partial (H_z \phi) \over \partial t} = mnR_{\phi} + {\partial \over
  \partial \sigma} \left( {K \over H_z} {\partial \phi \over \partial
  \sigma} \right)
\label{vdiff1} \end{equation} where $R_{\phi}$ represents all of the
forcing terms other than the vertical viscosity or diffusion.  Since we
want the diffusion term to be evaluated partly at the current timestep
$n$ and partly at the next timestep $n+1$, we introduce the parameter
$\lambda$ and rewrite equation (\ref{vdiff1}) as: \begin{equation}
  {\partial (H_z \phi) \over \partial t} = mnR_{\phi} + (1-\lambda)
  {\partial \over \partial \sigma} \left( {K \over H_z} {\partial \phi^n
  \over \partial \sigma} \right) + \lambda {\partial \over \partial
  \sigma} \left( {K \over H_z} {\partial \phi^{n+1} \over \partial \sigma}
  \right) .
\label{vdiff2} \end{equation} The discrete form of equation (\ref{vdiff2})
is: \begin{multline}
   {H_{z_k}^{n+1} \phi_k^{n+1} - H_{z_k}^n \phi_k^n \over \Delta t} = mn
   R_{\phi} + {(1-\lambda) \over \Delta s^2} \left[ {K_k \over H_{z_k}}
   (\phi_{k+1}^n - \phi_k^n) - {K_{k-1} \over H_{z_{k-1}}} (\phi_k^n -
   \phi_{k-1}^n) \right]
\\ \vspace{1mm}
 \qquad \qquad \qquad  + {\lambda \over \Delta s^2} \left[
   {K_k \over H_{z_k}} (\phi_{k+1}^{n+1} - \phi_k^{n+1}) - {K_{k-1}
   \over H_{z_{k-1}}} (\phi_k^{n+1} - \phi_{k-1}^{n+1}) \right]
\label{vdiff3} \end{multline} where $k$ is used as the vertical
level index.  This can be reorganized so that all the terms involving
$\phi^{n+1}$ are on the left and all the other terms are on the right.
The equation for $\phi_k^{n+1}$ will contain terms involving the neighbors
above and below ($\phi_{k+1}^{n+1}$ and $\phi_{k+1}^{n-1}$) which leads
to a set of coupled equations with boundary conditions for the top
and bottom.  The general form of these equations is: \begin{equation}
   A_k \phi_{k+1}^{n+1} + B_k \phi_k^{n+1} + C_k \phi_{k-1}^{n+1} = D_k
\end{equation} where the boundary conditions are written into the
coefficients for the end points.  In this case the coefficients become:
\begin{eqnarray}
   A(1) & = & 0 \\ A(2:{\bf N}) & = & - {\lambda \Delta t \, K_{k-1}
   \over \Delta \sigma^2 H^{n+1}_{z_{k-1}} } \\ B(1) & = & H^{n+1}_{z_1}
   + {\lambda \Delta t \, K_1 \over \Delta \sigma^2 H^{n+1}_{z_1} } \\
   B(2:{\bf Nm}) & = & H_{z_k}^{n+1} + {\lambda \Delta t \, K_k \over
   \Delta \sigma^2 H_{z_k}^{n+1} } + {\lambda \Delta t \, K_{k-1} \over
   \Delta \sigma^2 H^{n+1}_{z_{k-1}}}\\ B({\bf N}) & = & H^{n+1}_{z_{\bf
   N}} + {\lambda \Delta t \, K_{\bf Nm} \over
       \Delta \sigma^2 H^{n+1}_{z_{\bf Nm}} } \\
   C(1:{\bf Nm}) & = & - {\lambda \Delta t \, K_k \over \Delta \sigma^2
   H_{z_k}^{n+1} } \\ C({\bf N}) & = & 0 \\ D(1) & = & H^n_{z_1} \phi_1^n
   + \Delta t \, mn R_{{\phi}_1} + {\Delta t (1-\lambda) \over \Delta
   \sigma^2 }{ K_1 \over H^n_{z_1}} (\phi_2^n - \phi_1^n) - {\Delta t
   \over \Delta \sigma} \tau_b \\ D(2:{\bf Nm}) & = & H^n_{z_k} \phi_k^n +
   \Delta t \, mn R_{{\phi}_k} + \\ && {\Delta t (1-\lambda) \over \Delta
   \sigma^2 } \left[ { K_k \over H^n_{z_k}} (\phi_{k+1}^n - \phi_k^n)
   - { K_{k-1} \over H^n_{z_{k-1}}} (\phi_k^n - \phi_{k-1}^n) \right]
   \\ D({\bf N}) & = & H^n_{z_{\rm N}} \phi_{\rm N}^n + \Delta t \, mn
   R_{{\phi}_{\rm N}} - {\Delta t (1-\lambda) \over \Delta \sigma^2 } {
   K_{\bf Nm} \over H^n_{z_{\bf Nm}}} (\phi_{\rm N}^n - \phi_{\rm Nm}^n)
   + {\Delta t \over \Delta \sigma} \tau_s
\end{eqnarray} This is a standard tridiagonal system for which the
solution procedure can be found in any standard reference, such as
\citet{PFTV}.

