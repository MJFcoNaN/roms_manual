\section{Ocean Model Formulation}
\label{Phys}
\subsection {Equations of motion}
ROMS is a member of a general class of three-dimensional,
free-surface, terrain-following numerical models that solve the
Reynolds-averaged Navier-Stokes equations using the hydrostatic and
Boussinesq assumptions.
The governing equations in Cartesian coordinates can be written:
\begin{equation}
   \frac{\partial u}{\partial t} + \vec{v} \cdot \nabla u - fv =
   - \frac{\partial \phi}{\partial x}
   - \frac{\partial} {\partial z}
   \left( \overline{u'w'} - \nu \frac{\partial u}{\partial z} \right)
   + {\cal F}_u + {\cal D}_u
\label{st1}
\end{equation}
\begin{equation}
   \frac{\partial v}{\partial t} + \vec{v} \cdot \nabla v + fu =
   - \frac{\partial \phi}{\partial y}
   - \frac{\partial}{\partial z} \left( \overline{v'w'} -
   \nu \frac{\partial v}{\partial z} \right) 
    + {\cal F}_v + {\cal D}_v
\label{st2}
\end{equation}
\begin{equation}
 \frac{\partial \phi}{\partial z} = \frac{-\rho g}{\rho_o}
\label{st4}
\end{equation}
with the continuity equation:
\begin{equation}
  \frac{\partial u}{\partial x} +
  \frac{\partial v}{\partial y} + \frac{\partial w}{\partial z} = 0.
\label{st5}
\end{equation}
and scalar transport given by:
\begin{equation}
  \frac {\partial C}{\partial t} + \vec{v} \cdot \nabla C
  = - \frac{\partial}{\partial z} \left( \overline{C'w'} -
  \nu_\theta \frac{\partial C}{\partial z} \right) +
  {\cal F}_C + {\cal D}_C.
\label{st3a}
\end{equation}
An equation of state is also required:
\begin{equation}
   \rho = \rho(T,S,P)
\label{st3c}
\end{equation}
The variables are shown in Table \ref{ovars}.
An overbar represents a time average and a prime represents a
fluctuation about the mean. These equations are closed by
parameterizing the Reynolds stresses and turbulent tracer fluxes as:
\begin{equation}
  \overline{u'w'} = -K_M \frac{\partial u}{\partial z};\qquad
  \overline{v'w'} = -K_M \frac{\partial v}{\partial z};\qquad
  \overline{C'w'} = -K_C \frac{\partial C}{\partial z}.
\end{equation}
\begin{table}[thb]
\centerline{
\begin{tabular}{|c|l|} \hline
Variable & Description \\ \hline
  $C(x,y,z,t)$ & scalar quantity, i.e. temperature, salinity,
  nutrient concentration \\
  ${\cal D}_u, {\cal D}_v, {\cal D}_C$ & optional horizontal diffusive terms \\
  ${\cal F}_u, {\cal F}_v, {\cal F}_C$ & forcing/source terms \\
  $f(x,y)$ & Coriolis parameter \\
  $g$ & acceleration of gravity \\
  $h(x,y)$ & depth of sea floor below mean sea level \\
  $H_z(x,y,z)$ & vertical grid spacing \\
  $\nu, \nu_\theta$ & molecular viscosity and diffusivity \\
  $K_M, K_C$ & vertical eddy viscosity and diffusivity \\
  $P$ & total pressure $P \approx -\rho_o gz$ \\
  $\phi(x,y,z,t)$ & dynamic pressure $\phi = \left(P/\rho_o \right)$ \\
%  $p$ & pressure \\
%  $\rho(x,y,z,t), \rho_o$ & total and reference densities \\
  $\rho_o + \rho(x,y,z,t)$ & total {\sl in situ} density \\
  $S(x,y,z,t)$ & salinity \\
  $t$ & time \\
  $T(x,y,z,t)$ & potential temperature \\
  $u,v,w$ & the ($x,y,z$) components of vector velocity $\vec{v}$ \\
  % $u,v,\Omega$ & the ($x,y,z$) components of vector velocity $\vec{v}$ \\
  $x,y$ & horizontal coordinates \\
  $z$ & vertical coordinate \\
  $\zeta(x,y,t)$ & the surface elevation \\
  \hline
\end{tabular}
}
\label{ovars}
\caption{The variables used in the description of the ocean model}
\end{table}

Equations (\ref{st1}) and (\ref{st2}) express the momentum balance in
the $x$- and $y$-directions, respectively.  The time evolution of
all scalar concentration fields, including those for $T(x,y,z,t)$ and
$S(x,y,z,t)$, are governed by the advective-diffusive equation
(\ref{st3a}). The equation of state is given by
equation (\ref{st3c}).  In the Boussinesq approximation, density
variations are neglected in the momentum equations except for their
contribution to the buoyancy force in the vertical momentum equation
(\ref{st4}).  Under the hydrostatic approximation, it is further
assumed that the vertical pressure gradient balances the buoyancy
force.  Lastly, equation (\ref{st5}) expresses the continuity equation
for an incompressible fluid.  For the moment, the effects of forcing
and horizontal dissipation will be represented by the schematic terms
${\cal F}$ and ${\cal D}$, respectively.  The horizontal and vertical
mixing will be described more fully in \S\ref{Smooth}.

\subsection{Vertical boundary conditions}
\label{vbc}
The vertical boundary conditions can be prescribed as follows:
\[
\begin{array}{rl}
  \mbox{top ($z = \zeta(x,y,t))$} \hspace{1cm}
  & K_m \, \frac{\partial u}{\partial z}
    = \tau^x_s (x,y,t) \\ [1.5mm]
  & K_m \, \frac{\partial v}{\partial z} = \tau_s^y(x,y,t) \\[1.5mm]
  & K_C \, \frac{\partial C}{\partial z} = \frac{Q_C}{\rho_o c_P}
    \\[1.5mm]
  & w = \frac{\partial \zeta}{\partial t} \\[2mm]
  \mbox{and bottom ($z = -h(x,y)$)} \hspace{1cm} &
  K_m \, \frac{\partial u}{\partial z} = \tau_b^x (x,y,t) \\[1.5mm]
  & K_m \, \frac{\partial v}{\partial z} = \tau_b^y (x,y,t) \\[1.5mm]
  & K_C \, \frac{\partial C}{\partial z} = 0\\[1.5mm]
  & - w + \vec{v} \cdot \nabla h = 0.
\end{array}
\]

\begin{table}[h]
\centerline{
\begin{tabular}{|c|l|} \hline
Variable & Description \\ \hline
%  $E-P$ & evaporation minus precipitation \\
%   $\gamma_1, \gamma_2$ & linear and quadratic bottom stress
%   coefficients \\
  $Q_C$ & surface concentration flux \\
  $\tau_s^x , \tau_s^y$ & surface wind stress \\
  $\tau_b^x , \tau_b^y$ & bottom stress \\
  \hline
\end{tabular}
}
\label{vbcvars}
\caption{The variables used in the vertical boundary conditions for the
ocean model}
\end{table}

The surface boundary condition variables are defined in Table
\ref{vbcvars}.  Since $Q_T$ is a strong function of the surface
temperature, we usually choose to compute $Q_T$ using the surface
temperature and the atmospheric fields in an atmospheric bulk flux
parameterization. This bulk flux routine also computes the wind
stress from the winds.

On the variable bottom,
$z = -h(x,y)$, the horizontal velocity has
a prescribed bottom stress which is a choice between linear,
quadratic, or logarithmic terms.
% \begin{align*}
%   \tau_b^x & = (\gamma_1 + \gamma_2 \sqrt{u^2 + v^2} ) u + z_o blah \\
%   \tau_b^y & = (\gamma_1 + \gamma_2 \sqrt{u^2 + v^2} ) v + z_o blah
% \end{align*}
The vertical concentration flux may also be prescribed at the bottom,
although it is usually set to zero.

\subsection{Horizontal boundary conditions}
As distributed, the model can easily be configured for a periodic
channel, a doubly periodic domain, or a closed basin.  Code is also
included for open boundaries which may or may not work for your
particular application. Horizontal boundary conditions are
described more fully in \S\ref{HBC} and are
provided for $u,v,T,S,$ and $\zeta$.

The model domain is logically rectangular, but it is possible to
mask out land areas on the boundary and in the interior. Boundary
conditions on these masked regions are straightforward,
with a choice of no-slip or free-slip walls.

If biharmonic friction is used, a higher order boundary condition
must also be provided.  The model currently has this built into the
code where the biharmonic terms are calculated.  The high order
boundary conditions used for $u$ are $\frac{\partial}{\partial x} \left(
{\nu} \frac{\partial ^2 u}{\partial x^2} \right) = 0$ on the
eastern and western boundaries and $\frac{\partial}{\partial y} \left(
{\nu} \frac{\partial ^2 u}{\partial y^2} \right) = 0\,$ on the
northern and southern boundaries.  The boundary conditions for $v$
and $C$ are similar.  These boundary conditions were chosen because
they preserve the property of no gain or loss of volume-integrated
momentum or scalar concentration.

\subsection{Terrain-following coordinate system}
From the point of view of the computational model, it is highly
convenient to introduce a stretched vertical coordinate system which
essentially ``flattens out'' the variable bottom at $z = -h(x,y)$.
Such ``$\sigma$'' coordinate systems have long been used, with slight
appropriate modification, in both meteorology and oceanography
\citep[e.g.,][]{Phil,FHD}.
To proceed, we make the coordinate transformation:
\[\begin{array}{c}
  \hat{x} = x \\
  \hat{y} = y \\[1mm]
  \sigma = \sigma(x,y,z) \\[1mm]
  z = z(x,y,\sigma)
\end{array}\]
and
\[
  \hat{t} = t.
\]
See Appendix \ref{Scoord} for the form of $\sigma$ used here.
Also, see \citet{SS2005} for a
discussion about the nature of this form of $\sigma$ and how it
differs from that used in SCRUM.

In the stretched system, the vertical coordinate $\sigma$ spans the
range \mbox{$-1 \leq \sigma \leq 0$;} we are therefore left with
level upper ($\sigma = 0$) and lower ($\sigma = -1$) bounding
surfaces.  The chain rules for this transformation are:
\begin{gather*}
  \left( \frac{ \partial}{\partial x } \right)_z =
  \left( \frac{ \partial}{\partial x } \right)_\sigma -
  \left( \frac{ 1}{H_z } \right)
  \left( \frac{ \partial z}{\partial x } \right)_\sigma
  \frac{ \partial}{\partial \sigma}
\\
  \left( \frac{ \partial}{\partial y } \right)_z =
  \left( \frac{ \partial}{\partial y } \right)_\sigma -
  \left( \frac{ 1}{H_z } \right)
  \left( \frac{ \partial z}{\partial y } \right)_\sigma
  \frac{ \partial}{\partial \sigma}
\\
  \frac{ \partial}{\partial z } =
  \left( \frac{ \partial s}{\partial z } \right)
  \frac{ \partial}{\partial \sigma} =
  \frac{ 1}{H_z } \frac{ \partial}{\partial \sigma }
\end{gather*}
where
\[
  H_z \equiv \frac{ \partial z}{\partial \sigma }
\]
As a trade-off for this geometric
simplification, the dynamic equations become somewhat more
complicated.  The resulting dynamic equations are, after dropping the
carats:
\begin{equation}
  \frac{\partial u}{\partial t} - fv + \vec{v} \cdot \nabla u = - 
  \frac{\partial \phi}{\partial x} - \left( \frac{g\rho}
  {\rho_o} \right) \frac{\partial z}{\partial x} -
  g \frac{\partial \zeta}{\partial x} +
  \frac{ 1}{H_z } \frac{\partial}{\partial \sigma}
  \left[ \frac{K_m}{H_z} \frac{\partial u}{\partial \sigma} \right] +
  {\cal F}_u + {\cal D}_u
\label{st6}
\end{equation}
\begin{equation}
  \frac{\partial v}{\partial t} + fu + \vec{v} \cdot \nabla v = - 
  \frac{\partial \phi}{\partial y} - \left( \frac{g\rho}
  {\rho_o} \right) \frac{\partial z}{\partial y} -
  g \frac{\partial \zeta}{\partial y} +
  \frac{ 1}{H_z } \frac{\partial}{\partial \sigma}
  \left[ \frac{K_m}{H_z} \frac{\partial v}{\partial \sigma} \right] +
  {\cal F}_v + {\cal D}_v
\end{equation}
\begin{equation}
  \frac{\partial C}{\partial t} + \vec{v} \cdot \nabla C
  = \frac{ 1}{H_z } \frac{\partial}{\partial \sigma}
  \left[ \frac{K_C}{H_z} \frac{\partial C}{\partial \sigma} \right] +
  {\cal F}_{T} + {\cal D}_{T}
\end{equation}
\begin{equation}
  \rho = \rho(T,S,P)
\end{equation}
\begin{equation}
  \frac{\partial \phi}{\partial \sigma} = \left( \frac{-gH_z\rho}
  {\rho_o} \right)
\label{st9}
\end{equation}
\begin{equation}
  \frac{\partial H_z}{\partial t} +
  \frac{\partial (H_zu)}{\partial x} +
  \frac{\partial (H_zv)}{\partial y} +
  \frac{\partial (H_z \Omega)}{\partial \sigma} = 0
\label{st10}
\end{equation}
where
\[
  \vec{v} = (u,v,\Omega)
\]
\[
  \vec{v} \cdot \nabla = u \frac{\partial}{\partial x} + v
  \frac{\partial}{\partial y} + \Omega \frac{\partial}{\partial
  \sigma}.
\]
The vertical velocity in $\sigma$ coordinates is
\[
  \Omega (x,y,\sigma,t) = \frac{1}{H_z}
  \left[ w - \left(\frac{z+h}{\zeta+h} \right) 
  \frac{\partial \zeta}{\partial t} - u \frac{\partial z}{\partial x}
  - v \frac{\partial z}{\partial y} \right]
\]
and
\[
  w = \frac{\partial z}{\partial t} + u \frac{\partial z}{\partial x}
  + v \frac{\partial z}{\partial y} + \Omega H_z.
\]
In the stretched coordinate system, the vertical boundary conditions
become:
\[\begin{array}{rl}
  \mbox{top ($\sigma$= 0)} \hspace{1cm} & \left(\frac{K_m}{H_z}\right)
  \frac{\partial u}{\partial \sigma} = \tau^x_s (x,y,t) \\ [1.5mm]
  & \left(\frac{K_m}{H_z}\right) \frac{\partial v}{\partial \sigma}
  = \tau^y_s(x,y,t) \\[1.5mm]
  & \left(\frac{K_C}{H_z}\right) \frac{\partial C}{\partial \sigma}
  = \frac{Q_C}{\rho_o c_P}
  \\[1.5mm]
  & \Omega = 0 \\[2mm]
  \mbox{and bottom ($\sigma = -1$)} \hspace{1cm} &
  \left(\frac{K_m}{H_z}\right) \frac{\partial u}{\partial \sigma}
  = \tau^x_b (x,y,t) \\[1.5mm]
  & \left(\frac{K_m}{H_z}\right) \frac{\partial v}{\partial \sigma}
  = \tau^y_b (x,y,t) \\[1.5mm]
  & \left(\frac{K_C}{H_z}\right) \frac{\partial C}{\partial s}
  = 0 \\[1.5mm]
  & \Omega = 0.
\end{array}\]
Note the simplification of the boundary conditions on vertical
velocity that arises from the $\sigma$ coordinate transformation.

\subsection{Horizontal curvilinear coordinates}
\label{Ocurve}
In many applications of interest (e.g., flow adjacent to a coastal
boundary), the fluid may be confined horizontally within an
irregular region.  In such problems, a horizontal coordinate system
which conforms to the irregular lateral boundaries is advantageous.
It is often also true in many geophysical problems that the
simulated flow fields have regions of enhanced structure (e.g.,
boundary currents or fronts) which occupy a relatively small
fraction of the physical/computational domain.  In these problems,
added efficiency can be gained by placing more computational
resolution in such regions.

The requirement for a boundary-following coordinate system and for a
laterally variable grid resolution can both be met, for suitably
smooth domains, by introducing an appropriate orthogonal coordinate
transformation in the horizontal.  Let the new coordinates be $\xi
(x,y)$ and $\eta(x,y)$, where the relationship of horizontal arc
length to the differential distance is given by:
\begin{equation}
   (ds)_\xi = \left( \frac{1}{m} \right) d \xi
\end{equation}
\begin{equation}
   (ds)_\eta = \left( \frac{1}{n} \right) d \eta
\end{equation}
Here, $m(\xi,\eta)$ and $n(\xi,\eta)$ are the scale factors which
relate the differential distances $(\Delta \xi,\Delta \eta)$ to the
actual (physical) arc lengths.  Appendix \ref{Curve} contains the
curvilinear version of several common vector quantities.

Denoting the velocity components in the new coordinate system by
\begin{equation}
   \vec{v} \cdot \hat{\xi} = u
\end{equation}
and
\begin{equation}
   \vec{v} \cdot \hat{\eta} = v
\end{equation}
the equations of motion (\ref{st6})-(\ref{st10}) can be re-written
\citep[see, e.g.,][]{AL} as:
{\samepage
\begin{multline}
   \frac{\partial}{\partial t} \left( \frac{H_z u}{mn} \right) + \frac
   {\partial}{\partial \xi} \left( \frac{H_z u^2}{n} \right ) + \frac
   {\partial}{\partial \eta} \left( \frac{H_z uv}{m} \right) + \frac
   {\partial}{\partial \sigma} \left( \frac{H_z u\Omega}{mn} \right)
\\
   - \left\{\left(\frac{f}{mn} \right) + v \frac{\partial}{\partial \xi}
   \left( \frac{1}{n} \right) - u \frac{\partial}{\partial \eta} \left(
   \frac{1}{m} \right) \right\} H_z v =
\\
   -\left( \frac{H_z }{n} \right )
   \left( \frac{\partial \phi}{\partial \xi} +
   \frac{g \rho}{\rho_o} \frac{\partial z}{\partial \xi} +
   g \frac{\partial \zeta}{\partial \xi} \right) +
   \frac{ 1}{mn} \frac{\partial}{\partial \sigma}
   \left[ \frac{K_m}{H_z} \frac{\partial u}{\partial \sigma} \right] +
   \frac{ H_z}{mn}
   \left( {\cal F}_u + {\cal D}_u \right)
\label{st13}
\end{multline}
}
{\samepage
\begin{multline}
   \frac{\partial}{\partial t} \left( \frac{H_z v}{mn} \right) + \frac
   {\partial}{\partial \xi} \left( \frac{H_z uv}{n} \right ) + \frac
   {\partial}{\partial \eta} \left( \frac{H_z v^2}{m} \right) + \frac
   {\partial}{\partial \sigma} \left( \frac{H_z v\Omega}{mn} \right)
\\
   + \left\{\left(\frac{f}{mn} \right) + v \frac{\partial}{\partial \xi}
   \left( \frac{1}{n} \right) - u \frac{\partial}{\partial \eta} \left(
   \frac{1}{m} \right) \right\} H_z u =
\\
   -\left( \frac{H_z }{m} \right )
   \left( \frac{\partial \phi}{\partial \eta} +
   \frac{g \rho}{\rho_o} \frac{\partial z}{\partial \eta} +
   g \frac{\partial \zeta}{\partial \eta} \right) +
   \frac{ 1}{mn} \frac{\partial}{\partial \sigma}
   \left[ \frac{K_m}{H_z} {\partial v}{\partial \sigma} \right] +
   \frac{ H_z}{mn}
   \left( {\cal F}_v + {\cal D}_v \right)
\label{st14}
\end{multline}
}
\begin{multline}
   \frac{\partial}{\partial t} \left( \frac{H_z C}{mn} \right) +
   \frac {\partial}{\partial \xi} \left( \frac{H_z uC}{n} \right ) +
   \frac {\partial}{\partial \eta} \left( \frac{H_z vC}{m} \right) +
   \frac {\partial}{\partial \sigma}
   \left( \frac{H_z \Omega C}{mn} \right) =
\\
   \frac{ 1}{mn} \frac{\partial}{\partial s}
   \left[ \frac{K_C}{H_z} \frac{\partial C}{\partial \sigma} \right] +
   \frac{ H_z}{mn}
   \left( {\cal F}_{C} + {\cal D}_{C} \right)
\label{st15}
\end{multline}
\begin{equation}
  \rho = \rho(T,S,P)
\end{equation}
\begin{equation}
   \frac{\partial \phi}{\partial \sigma} = -\left( \frac{gH_z \rho}
   {\rho_o} \right)
\label{st16}
\end{equation}
\begin{equation}
   \frac{\partial}{\partial t} \left( \frac{H_z}{mn} \right) +
   \frac{\partial}{\partial \xi} \left( \frac{H_z u}{n} \right) +
   \frac{\partial}{\partial \eta} \left( \frac{H_z v}{m} \right) +
   \frac{\partial}{\partial \sigma}\left( \frac{H_z \Omega}{mn} \right)
   = 0.
\label{st17}
\end{equation}
All boundary conditions remain unchanged.
