\section{\code{Makefiles}}
\label{Gmake}
One of the software development tools which comes with the UNIX
operating system is called \code{make}. \code{Make} has many uses,
but is most commonly used to keep track of how a large program
should be compiled. ROMS now requires the \code{gnu} version of 
\code{make}, sometimes known as \code{gmake} \citep{GMAKE}.
This appendix describes generic \code{make}, the \code{gnu}
enhancements to \code{make}, as well as describing the
\code{makefile} structure used in ROMS. This material first
appeared in the \href{http://www.arsc.edu/support/news/HPCnews.shtml}{ARSC
HPC Newsletter} in several segments and has been updated and restructured
here.

\subsection{Introduction to Portable \code{make}}

\code{Make} gets its instructions from a description file, by default named
\code{makefile} or \code{Makefile}. This file is called the
\code{Makefile}, but other files
can be used by invoking \code{make} with the \code{-f} option:
\begin{verbatim}
     make -f Makefile.yukon
\end{verbatim}

The ancester to ROMS originally had a \code{Makefile} that looked
something like:
\begin{verbatim}
model: main.o init.o plot.o
<TAB>   f77 -o model main.o init.o plot.o

main.o: main.f
<TAB>   f77 -c -O main.f

init.o: init.f
<TAB>   f77 -c -O init.f

plot.o: plot.f
<TAB>   f77 -c -O0 plot.f

clean:
<TAB>   rm *.o core
\end{verbatim}
The default thing to build is \code{model}, the first target. The syntax
is:
\begin{verbatim}
target: dependencies
<TAB>  command
<TAB>  command
\end{verbatim}
The target \code{model} depends on the object files, \code{main.o}
and friends. They have to exist and be up to date before model's link
command can be run. If the target is out of date according to the
timestamp on the file, then the commands will
be run. Note that the tab is required on the command lines.

The other targets tell \code{make} how to create the object files
and that they in turn have dependencies.  To compile \code{model}, simply
type ``\code{make}''. \code{Make} will look for the file \code{makefile},
read it, and do whatever is necessary to make \code{model} up
to date. If you edit \code{init.f}, that file will be newer than
\code{init.o}. \code{Make} would see that \code{init.o} is out of date
and run the ``\code{f77 -c -O init.f}'' command. Now \code{init.o}
is newer than \code{model}, so the link command ``\code{f77 -o model
main.o init.o plot.o}'' must be executed.

To clean up, type ``\code{make clean}'' and the \code{clean} target will
be brought up to date. The \code{clean} target has no dependencies. What
happens in that case is that the command (``\code{rm *.o core}'') will
always be executed.

The original version of this \code{Makefile} turned off
optimization on \code{plot.f} due to a compiler bug, but hopefully you
won't ever have to worry about that.

\subsubsection{Macros}

\code{Make} supports a simple string substitution macro. Set it with:
\begin{verbatim}
MY_MACRO = nothing today
\end{verbatim}
and refer to it with:
\begin{verbatim}
$(MY_MACRO)
\end{verbatim}

The convention is to put the macros near the top of your \code{Makefile}
and to use upper case. Also, use separate macros for the name of
your compiler and the flags it needs:
\begin{verbatim}
F90 = f90
F90FLAGS = -O3
LIBDIR = /usr/local/lib
LIBS = -L$(LIBDIR) -lmylib
\end{verbatim}

Let's rewrite our \code{Makefile} using macros:
\begin{verbatim}
#
# IBM version
#
F90 = xlf90_r
F90FLAGS = -O3 -qstrict
LDFLAGS = -bmaxdata:0x40000000

model: main.o init.o plot.o
        $(F90) $(LDFLAGS) -o model main.o init.o plot.o

main.o: main.f
        $(F90) -c $(F90FLAGS) main.f

init.o: init.f
        $(F90) -c $(F90FLAGS) init.f

plot.o: plot.f
        $(F90) -c $(F90FLAGS) plot.f

clean:
        rm *.o core
\end{verbatim}

Now when we change computers, we only have to change the compiler name
in one place. Likewise, if we want to try a different optimization level,
we only specify that in one place.

Notice that you can use comments by starting the line with a \code{\#}.

\subsubsection{Implicit Rules}

\code{Make} has some rules already built in. For fortran, you might be able
to get away with:
\begin{verbatim}
OBJS = main.o init.o plot.o

model: $(OBJS)
        $(FC) $(LDFLAGS) -o model $(OBJS)
\end{verbatim}
as your whole \code{Makefile}. \code{Make} will automatically invoke its
default Fortran compiler, possibly \code{f77} or \code{g77}, with whatever
default compile options it happens to have (\code{FFLAGS}). One built
in rule often looks like:
\begin{verbatim}
.c.o:
        $(CC) $(CFLAGS) -c $<
\end{verbatim}
which says to compile \code{.c} files to \code{.o} files using the
compiler \code{CC} and options \code{CFLAGS}. We can write our own suffix
rules in this same style.  The only thing to watch for is that make by
default has a limited set of file extentions that it knows about. Let's
write our \code{Makefile} using a suffix rule:
\begin{verbatim}
#
# Cray version
#
F90 = f90
F90FLAGS = -O3
LDFLAGS =

.f.o:
        $(F90) $(F90FLAGS) -c $<

model: main.o init.o plot.o
        $(F90) $(LDFLAGS) -o model main.o init.o plot.o

clean:
        rm *.o core
\end{verbatim}

\subsubsection{Dependencies}

There may be additional dependencies beyond the \code{source->object} ones.
In our little example, all our source files include a file called
\code{commons.h}. If \code{commons.h} gets modified to add a new variable,
everything must be recompiled. \code{Make} won't know that unless you
tell it:
\begin{verbatim}
# include dependencies
main.o: commons.h
init.o: commons.h
plot.o: commons.h
\end{verbatim}
Fortran 90 introduces module dependencies as well. See \S\ref{sfm}
for how we automatically generate this dependency information.

The most common newbie mistake is to forget that the commands after
a target {\em have} to start with a tab.

\subsection{\code{gnu make}}

Over the years, the community has moved from the stance of writing
portable \code{Makefiles} to a stance of just using a powerful, portable
\code{make}. The previous section described a portable subset of
\code{make} features. We now delve into some of the more powerful
tools available in \code{gnu make}.

\subsubsection{Make rules}

The core of \code{make} hasn't changed in decades, but concentrating on
\code{gmake} allows one to make use of its nifty little extras designed
by real programmers to help with real projects. The first change that
matters to my \code{Makefiles} is the change from suffix rules to
pattern rules. I've always found the \code{.SUFFIXES} list to be odd,
especially since \code{.f90} is not in the default list. Good riddance
to all of that! For a concrete example, the old way to provide a rule
for going from \code{file.f90} to \code{file.o} is:
\begin{verbatim}
.SUFFIXES: .o .f90 .F .F90
.f90.o:
<TAB>   $(FC) -c $(FFLAGS) $<
\end{verbatim}
while the new way is:
\begin{verbatim}
%.o: %.f90
<TAB>   $(FC) -c $(FFLAGS) $<
\end{verbatim}
In fact, going to a uniform \code{make} means that we can figure out what
symbols are defined and use their standard values---in this case,
\$\code{(FC)} and \$\code{(FFLAGS)} are the built-in default names
for the compiler and its options.
If you have any
questions about this, you can always run \code{make} with the \code{-p} (or
\code{-\mbox{}-print-data-base}) option. This prints out all the rules
\code{make} knows about, such as:
\begin{verbatim}
# default
COMPILE.f = $(FC) $(FFLAGS) $(TARGET_ARCH) -c
\end{verbatim}
Printing the rules database will show variables that \code{make} is
picking up from the environment, from the \code{Makefile}, and from its
built-in rules---and which of these sources is providing each
value.

\subsubsection{Assignments}

In the old days, I only used one kind of assignment: \code{=} (equals
sign). \code{Gmake} has several kinds of assignment (other \code{makes}
might as well, but I no longer have to know or care). An example of
the power of \code{gnu make} is shown by an example from my Cray X1
\code{Makefile}.  There is a routine which runs much more quickly if a
short function in another file is inlined. The way to accomplish this is
through the \code{-O inlinefrom=file} directive to the compiler. I can't
add this option to \code{FFLAGS}, since the inlined routine won't compile
with this directive---there is only the one file that needs it. I had:
\begin{verbatim}
    FFLAGS = -O 3,aggress -e I -e m
    FFLAGS2 = -O 3,aggress -O inlinefrom=lmd_wscale.f90 -e I -e m

lmd_skpp.o:
<TAB>   $(FC) -c $(FFLAGS2) $*.f90
\end{verbatim}
The first change I can make to this using other assignments is:
\begin{verbatim}
    FFLAGS := -O 3,aggress -e I -e m
    FFLAGS2 := $(FFLAGS) -O inlinefrom=lmd_wscale.f90
\end{verbatim}
The \code{:=} assignment means to evaluate the right hand side immediately.
In this case, there is no reason not to, as long as the second
assigment follows the first one (since it's using the value of
\$\code{(FFLAGS))}. For the plain equals, \code{make} doesn't evaluate the
right-hand side until its second pass through the \code{Makefile}. However,
\code{gnu make}
supports an assignment which avoids the need for \code{FFLAGS2} entirely:
\begin{verbatim}
lmd_skpp.o: FFLAGS += -O inlinefrom=lmd_wscale.f90
\end{verbatim}
What this means is that for the target \code{lmd\_skpp.o} only, append the
inlining directive to \code{FFLAGS}. I think this is pretty cool!

One last kind of assignment is to set the value only if there is no
value from somewhere else (the environment, for instance):
\begin{verbatim}
     FC ?= mpxlf90_r
\end{verbatim}
If we used \code{:=} or \code{=}, we would override the value from the
environment.

\subsubsection{Include and a Few Functions}
\label{Inc_fort}

One reasonably portable \code{make} feature is the include directive. It
can be used to clean up the \code{Makefile} by putting bits in an
include file. The syntax is simply:
\begin{verbatim}
   include file
\end{verbatim}
and we use it liberally to keep the project
information neat. We can also include a file with the system/compiler
information in it, assuming we have some way of deciding {\em which}
file to include. We can use \code{uname -s} to find out which operating
system we're using. We also need to know which compiler we're using.

One user-defined variable is called \code{FORT}, the name of the
Fortran compiler. This value is combined with the result of
``\code{uname -s}'' to
provide a machine and compiler combination. For instance, \code{ftn} on
Linux is the Cray cross-compiler. This would link to a different copy
of the NetCDF library and use different compiler flags than the Intel
compiler which might be on the same system.
\begin{verbatim}
# The user sets FORT:
  FORT ?= ftn
  OS := $(shell uname -s | sed 's/[\/ ]/-/g')

  include $(COMPILERS)/$(OS)-$(strip $(FORT)).mk
\end{verbatim}
We pick one include file at compile time, here picking
\code{Linux-ftn.mk}, containing the Cray cross-compiler information. The
value \code{Linux} comes from the \code{uname} command, the \code{ftn}
comes from the user, and the two are concatenated.
The sed command will turn the
slash in \code{UNICOS/mp} into a dash; the native Cray include file is
\code{UNICOS-mp-ftn.mk}. Strip is a built-in function to strip away
any extra white space.

Another tricky system is \code{CYGWIN}, which puts a version number
in the \code{uname} output, such as \code{CYGWIN\_NT-5.1}. \code{Gnu make} has
quite a few built-in functions, one of which allows us to do string
substitution:
\begin{verbatim}
OS := $(patsubst CYGWIN_%,CYGWIN,$(OS))
\end{verbatim}
In \code{make}, the \code{\%} symbol is a sort of wild card, much
like \code{*} in the shell.
Here, we match \code{CYGWIN} followed by an underscore and anything else,
replacing the whole with simply \code{CYGWIN}. Another example of a
built-in function is the substitution we do in:
\begin{verbatim}
   objects = $(subst .F,.o,$(sources))
\end{verbatim}
where we build our list of objects from the list of sources.
There are quite a few other functions, plus the user can define
their own. From \citet{GMAKE}:
\begin{quote}
GNU make supports both built-in and user-defined functions.
A function invocation looks much like a variable reference, but
includes one or more parameters separated by commas.  Most built-in
functions expand to some value that is then assigned to a variable
or passed to a subshell. A user-defined function is stored in a
variable or macro and expects one or more parameters to be passed
by the caller.
\end{quote}
We will show some user-defined functions in \S\ref{make_fun2}.

\subsubsection{Conditionals}

We used to have way too many \code{Makefiles}, a separate one for each
of the serial/MPI/OpenMP versions on each system (if supported). For
instance, the name of the IBM compiler changes when using MPI; the
options change for OpenMP. The compiler options also change when using
64-bit addressing or for debugging.
A better way to organize this is to have the user select 64-bit or not, MPI
or not, etc., then use conditionals. The complete list of
user definable \code{make} variables is given in \S\ref{make_var}.

\code{Gnu make} supports two kinds of \code{if} test, \code{ifdef} and
\code{ifeq} (plus the
negative versions \code{ifndef, ifneq}). The example from the book is:
\begin{verbatim}
ifdef COMSPEC
   # We are running Windows
else
   # We are not on Windows
endif
\end{verbatim}
An example from the IBM include file is:
\begin{verbatim}
ifdef USE_DEBUG
           FFLAGS += -g -qfullpath
else
           FFLAGS += -O3 -qstrict
endif
\end{verbatim}
To test for equality, an example is:
\begin{verbatim}
ifeq ($(USE_MPI),on)
   # Do MPI things
endif
\end{verbatim}
or 
\begin{verbatim}
ifeq "$(USE_MPI)" "on"
   # Do MPI things
endif
\end{verbatim}
The user has to set values for the \code{USE\_MPI}, \code{USE\_DEBUG},
and \code{USE\_LARGE} switches in the \code{Makefile} or the build
script {\em before} the compiler-dependent piece is included:
\begin{verbatim}
    USE_MPI ?= on
    USE_DEBUG ?=
    USE_LARGE ?=
\end{verbatim}
The \code{Makefile} uses the conditional assign ``\code{?=}''
in case a build script is used to set them in the environment.
Be sure to leave the switches meant to be off set to an empty string---the
string ``\code{off}'' will test true on an \code{ifdef} test.

\subsection{Multiple Source Directories the ROMS Way}
\label{Mult_dir_make}

There's more than one way to divide your sources into separate
directories. The choices we have made include nonrecursive \code{make}
and putting the temporary files in their own \code{\$(SCRATCH\_DIR)}
directory. These include the \code{.f90} files which have been through
the C preprocessor, object files, module files, and libraries.

\subsubsection{Directory Structure}

The directory structure of the source code has the top directory,
a \code{Master} directory, a \code{ROMS} directory with a number of
subdirectories, and several other directories. \code{Master} contains
the main program while the rest contain sources for libraries and
other files. Note that the bulk of the source code gets compiled into
files that become libraries with the \code{ar} command, one library per
directory. There is also a \code{Compilers} directory for the system-
and compiler-specific \code{Makefile} components.

\subsubsection{Conditionally Including Components}

The \code{makefile} will build the lists of libraries to create and source
files to compile. They start out empty at the top of the
\code{makefile}:
\begin{verbatim}
  sources    := 
  libraries  :=
\end{verbatim}
That's simple enough, but the list of directories to search for
these sources will depend on the options chosen by the user, not
just in the \code{make} options (\S\ref{make_var}), but inside the
\code{ROMS\_HEADER} file as well. How does this happen? Once \code{make}
knows how to find the \code{ROMS\_HEADER}, it is used by \code{cpp}
to generate an include file telling \code{make} about these other options.
\begin{verbatim}
MAKE_MACROS := Compilers/make_macros.mk
 MACROS := $(shell cpp -P $(ROMS_CPPFLAGS) Compilers/make_macros.h > \
                 $(MAKE_MACROS); $(CLEAN) $(MAKE_MACROS))
\end{verbatim}
The \code{make\_macros.h} file contains blocks such as:
\begin{verbatim}
#ifdef SWAN_COUPLING
  USE_SWAN := on
#else
  USE_SWAN :=
#endif
\end{verbatim}
The resulting \code{make\_macros.mk} file will simply end up with
either 
\begin{verbatim}
  USE_SWAN := on
\end{verbatim}
or
\begin{verbatim}
  USE_SWAN :=
\end{verbatim}
This file can then be included by the \code{makefile} and the
variable \code{USE\_SWAN} will have the correct state for this
particular compilation. We can now use it and all the similar flags
to build a list of directories.

We initialize two lists:
\begin{verbatim}
 modules  :=
 includes :=    ROMS/Include
\end{verbatim}
Add the adjoint bits:
\begin{verbatim}
ifdef USE_ADJOINT
 modules  +=    ROMS/Adjoint
endif
ifdef USE_ADJOINT
 includes +=    ROMS/Adjoint
endif
\end{verbatim}
Add the bits we'll always need:
\begin{verbatim}
 modules  +=    ROMS/Nonlinear \
                ROMS/Functionals \
                ROMS/Utility \
                ROMS/Modules
 includes +=    ROMS/Nonlinear \
                ROMS/Utility \
                ROMS/Drivers
\end{verbatim}
Then we add in some more:
\begin{verbatim}
ifdef USE_SWAN
 modules  +=    Waves/SWAN/Src
 includes +=    Waves/SWAN/Src
endif

 modules  +=    Master
 includes +=    Master Compilers
\end{verbatim}
Now that our lists are complete, let's put them to use:
\begin{verbatim}
vpath %.F $(modules)
vpath %.h $(includes)
vpath %.f90 $(SCRATCH_DIR)
vpath %.o $(SCRATCH_DIR)

include $(addsuffix /Module.mk,$(modules))

CPPFLAGS += $(patsubst %,-I%,$(includes))
\end{verbatim}
\begin{enumerate}
\item \code{vpath} is a standard \code{make} feature for
providing a list of directories for \code{make} to search for files
of different types. Here we are saying that \code{*.F} files can be
found in the directories provided in the \code{\$(modules)} list, and
so on for the others.

\item For each directory in the \code{\$(modules)} list, \code{make}
will include the file \code{Module.mk} that is found there. More on
these later.

\item For each directory in the \code{\$(includes)} list, add
that directory to the list searched by \code{cpp} with the \code{$-$I}
flag.
\end{enumerate}

\subsubsection{User-defined \code{make} Functions}
\label{make_fun2}

The \code{Module.mk} fragments mentioned before call some
user-defined functions.  It's time to show these functions and
talk about how they work. They get defined in the top Makefile:
\begin{verbatim}
#--------------------------------------------------------------------------
#  Make functions for putting the temporary files in $(SCRATCH_DIR)
#--------------------------------------------------------------------------

# $(call source-dir-to-binary-dir, directory-list)
source-dir-to-binary-dir = $(addprefix $(SCRATCH_DIR)/, $(notdir $1))

# $(call source-to-object, source-file-list)
source-to-object = $(call source-dir-to-binary-dir,   \
                   $(subst .F,.o,$(filter %.F,$1)))

# $(call f90-source, source-file-list)
f90-source = $(call source-dir-to-binary-dir,     \
                   $(subst .F,.f90,$1))

# $(call make-library, library-name, source-file-list)
define make-library
   libraries += $(SCRATCH_DIR)/$1
   sources   += $2

   $(SCRATCH_DIR)/$1: $(call source-dir-to-binary-dir,    \
                      $(subst .F,.o,$2))
       $(AR) $(ARFLAGS) $$@ $$^
       $(RANLIB) $$@
endef

# $(call one-compile-rule, binary-file, f90-file, source-files)
define one-compile-rule
  $1: $2 $3
       cd $$(SCRATCH_DIR); $$(FC) -c $$(FFLAGS) $(notdir $2)

  $2: $3
       $$(CPP) $$(CPPFLAGS) $$(MY_CPP_FLAGS) $$< > $$@
       $$(CLEAN) $$@

endef

# $(compile-rules)
define compile-rules
  $(foreach f, $(local_src),       \
    $(call one-compile-rule,$(call source-to-object,$f), \
    $(call f90-source,$f),$f))
endef
\end{verbatim}

\begin{enumerate}

\item We define a function to convert the path from
the source directory to the \code{Build} directory, called
\code{source-dir-to-binary-dir}. Note that the \code{Build} directory
is called \$\code{(SCRATCH\_DIR)} here. All it does is strip off the
leading directory with the the built-in function \code{notdir}, then
paste on the \code{Build} directory.

\item Next comes \code{source-to-object}, which calls the function above to
return the object filename when given the source filename. It assumes
that all sources have a \code{.F} extension.

\item A similar function is \code{f90-source}, which returns the name of the
intermediate source which is created by \code{cpp} from our
\code{.F} file.

\item The \code{Module.mk} fragment in each library source directory invokes
\code{make-library}, which takes the library name and the list of sources
as its arguments. The function adds its \code{library} to the global list
of \code{libraries} and provides rules for building itself. The double
dollar signs are to delay the variable substitution. Note that we call
\code{source-dir-to-binary-dir} instead of \code{source-to-object}---this
is a work-around for a make bug.

\item The next, \code{one-compile-rule}, takes three arguments: the \code{.o}
filename, the \code{.f90} filename, and the \code{.F} filename. From
these, it produces the \code{make} rules for running \code{cpp} and the
compiler.

A note on directories: \code{make} uses \code{vpath} to find the
source file where it resides. It would be possible to compile from
the top directory and put the \code{.o} file in \code{Build} with the
appropriate arguments, but I don't know how to get the \code{.mod} file
into \code{Build} short of a \code{mv} command. Likewise, if we compile
in the top directory, we need to know the compiler option to tell it to
look in \code{Build} for the \code{.mod} files it uses. Doing a \code{cd}
to \code{Build} before compiling is just simpler.

\item The last, \code{compile-rules}, is given a list of sources, then calls
\code{one-compile-rule} once per source file.
\end{enumerate}

Again, you can invoke \code{make -p} to see how \code{make}
internally transforms all this into actual targets and rules.

\subsubsection{Library \code{Module.mk}}

In each library directory, there is a file called \code{Module.mk}
which gets included by the top level \code{makefile}. These
\code{Module.mk} bits build onto the list of sources and libraries
to be compiled and built, respectively.
These \code{Module.mk} files look something like:
\begin{verbatim}
local_sub  := ROMS/Nonlinear
local_lib  := libNLM.a

local_src  := $(wildcard $(local_sub)/*.F)

$(eval $(call make-library,$(local_lib),$(local_src)))

$(eval $(compile-rules))
\end{verbatim}
First, we provide the name of the current directory and the library
to be built from the resident sources.  Next, we use the \code{wildcard}
function to search the subdirectory for these sources. Note that
every \code{.F} file found will be compiled. If you have half-baked
files that you don't want used, make sure they have a different
extension.

Each subdirectory is resetting the \code{local\_src} variable. That's
OK because we're saving the values in the global \code{sources} variable
inside the \code{make-library} function, which also adds the local library
to the \code{libraries} list. The \code{compile-rules} function uses this
\code{local\_src} variable to generate the rules for compiling each file,
placing the resulting files in the \code{Build} directory.

%the user-defined \code{subdirectory} function, from \citet{GMAKE}:
%\begin{verbatim}
%subdirectory  = $(patsubst %/Module.mk,%, \
%                $(word $(words $(MAKEFILE_LIST)),$(MAKEFILE_LIST)))
%\end{verbatim}
%This does the right thing to figure out which subdirectory we are
%in from \code{make}'s internal list of the \code{Makefiles} it is
%parsing. It depends on all the subdirectory include files being called
%\code{Module.mk}.

\subsubsection{Main Program}

The main program is in a directory called \code{Master} and its
\code{Module.mk} is similar to the library one:
\begin{verbatim}
local_sub  := Master
local_src  := $(wildcard $(local_sub)/*.F)

local_objs := $(subst .F,.o,$(local_src))
local_objs := $(addprefix $(SCRATCH_DIR)/, $(notdir $(local_objs)))

sources    += $(local_src)

ifdef LD_WINDOWS
$(BIN): $(libraries) $(local_objs)
        $(LD) $(FFLAGS) $(local_objs) -o $@ $(libraries) $(LIBS_WIN32) $(LDFLAGS)
else
$(BIN): $(libraries) $(local_objs)
        $(LD) $(FFLAGS) $(LDFLAGS) $(local_objs) -o $@ $(libraries) $(LIBS)
endif

$(eval $(compile-rules))
\end{verbatim}
Instead of a rule for building a library, we have a rule for building
a binary \code{\$(BIN)}. In this case, the name of the ROMS binary
is defined elsewhere. The binary depends
on the \code{libraries} getting compiled first, as well as the local
sources. During the link, the \$\code{(libraries)} are compiled from
the sources in the other directories, while \$\code{(LIBS)} are exteral
libraries such as NetCDF.

\subsubsection{Top Level Makefile}

Now we get to the glue that holds it all together. We've covered
many things so far, but there's still a few bits which might be
confusing:
\begin{enumerate}
\item There can be rare cases where you might have special code for
some systems. You can check which system you are on in the \code{.F}
file with:
\begin{verbatim}
#ifdef X86_64
!      special stuff
#endif
\end{verbatim}
To be sure this is defined on each \code{X86\_64} system, it has to
be passed to \code{cpp}:
\begin{verbatim}
CPPFLAGS += -D$(shell echo ${OS} | tr "-" "_" | tr [a-z] [A-Z])
CPPFLAGS += -D$(shell echo ${CPU} | tr "-" "_" | tr [a-z] [A-Z])
CPPFLAGS += -D$(shell echo ${FORT} | tr "-" "_" | tr [a-z] [A-Z])

CPPFLAGS += -D'ROOT_DIR="$(ROOTDIR)"'
ifdef ROMS_APPLICATION
  CPPFLAGS  += $(ROMS_CPPFLAGS)
  CPPFLAGS  += -DNestedGrids=$(NestedGrids)
  MDEPFLAGS += -DROMS_HEADER="$(HEADER)"
endif
\end{verbatim}
This guarantees that \code{CPPFLAGS} will have terms in it such
as:
\begin{verbatim}
       -DLINUX -DX86_64 -DPGI
       -D'ROOT_DIR="/export/staffdata/kate/roms/trunk"' -DSHOREFACE
       -D'HEADER="shoreface.h"' -D'ROMS_HEADER="shoreface.h"'
       -DNestedGrids=1
\end{verbatim}

\item For \code{mod\_strings.F}, there is a special case:
\begin{verbatim}
$(SCRATCH_DIR)/mod_strings.f90: CPPFLAGS += -DMY_OS='"$(OS)"' \
              -DMY_CPU='"$(CPU)"' -DMY_FORT='"$(FORT)"' \
              -DMY_FC='"$(FC)"' -DMY_FFLAGS='"$(FFLAGS)"'
\end{verbatim}
allowing ROMS to report in its output:
\begin{verbatim}
 Operating system : Linux
 CPU/hardware     : x86_64
 Compiler system  : pgi
 Compiler command : pgf90
 Compiler flags   :  -O3 -tp k8-64 -Mfree

 Local Root    : /export/staffdata/kate/roms/trunk
 Header Dir    : ./ROMS/Include
 Header file   : shoreface.h
\end{verbatim}
Though this doesn't seem to work on the Mac.

\item The very first \code{makefile} I showed had a set list of
files to remove on \code{make clean}. We now build a list,
called \code{clean\_list}:
\begin{verbatim}
 clean_list := core *.ipo $(SCRATCH_DIR)

ifeq "$(strip $(SCRATCH_DIR))" "."
  clean_list := core *.o *.oo *.mod *.f90 lib*.a *.bak
  clean_list += $(CURDIR)/*.ipo
endif
\end{verbatim}
In other words, we want to clean up the \code{Build} directory unless
it happens to be the top level directory, in which case we only want
to remove specific files there.

\item ``\code{all}'' is the first target that gets seen by \code{make},
making it the default \code{target}. In this case, we know there is only
the one binary, whose name we know---\citet{GMAKE} shows what to
do with more than one binary. Both ``\code{all}'' and ``\code{clean}''
are phony targets in that no files of those names get generated---make
has the \code{.PHONY} designation for such targets. Also, the \code{clean}
target doesn't require any compiler information, so the compiler include
doesn't happen if the target is ``clean'':

\begin{verbatim}
ifneq "$(MAKECMDGOALS)" "clean"
  include $(COMPILERS)/$(OS)-$(strip $(FORT)).mk
endif
\end{verbatim}
\code{\$(MAKECMDGOALS)} is a special variable containing the current
\code{make} target.

\item We'll be creating different executable names, depending on
which options we've picked:
\begin{verbatim}
BIN := $(BINDIR)/oceanS
ifdef USE_DEBUG
  BIN := $(BINDIR)/oceanG
else
 ifdef USE_MPI
   BIN := $(BINDIR)/oceanM
 endif
 ifdef USE_OpenMP
   BIN := $(BINDIR)/oceanO
 endif
endif
\end{verbatim}

\item The NetCDF library gets included during the final link stage.
However, we are now using the Fortran 90 version of it which
requires its module information as well. We just copy the \code{.mod}
files into the \code{Build} directory:
\begin{verbatim}
   NETCDF_MODFILE := netcdf.mod
TYPESIZES_MODFILE := typesizes.mod

$(SCRATCH_DIR)/$(NETCDF_MODFILE): | $(SCRATCH_DIR)
        cp -f $(NETCDF_INCDIR)/$(NETCDF_MODFILE) $(SCRATCH_DIR)

$(SCRATCH_DIR)/$(TYPESIZES_MODFILE): | $(SCRATCH_DIR)
        cp -f $(NETCDF_INCDIR)/$(TYPESIZES_MODFILE) $(SCRATCH_DIR)
\end{verbatim}
Old versions of NetCDF do not have the \code{typesizes.mod} file, in which
case it has to be removed from the following dependency list:
\begin{verbatim}
$(SCRATCH_DIR)/MakeDepend: makefile \
                           $(SCRATCH_DIR)/$(NETCDF_MODFILE) \
                           $(SCRATCH_DIR)/$(TYPESIZES_MODFILE) \
                           | $(SCRATCH_DIR)
\end{verbatim}

\item Then there is \code{MakeDepend} itself. This file is created
by the \code{Perl} script \code{sfmakedepend}. \code{MakeDepend}
gets created by ``\code{make depend}'' and included on \code{make}'s
second pass through the \code{makefile}:
\begin{verbatim}
depend: $(SCRATCH_DIR)
        $(SFMAKEDEPEND) $(MDEPFLAGS) $(sources) > $(SCRATCH_DIR)/MakeDepend 

ifneq "$(MAKECMDGOALS)" "clean"
  -include $(SCRATCH_DIR)/MakeDepend
endif
\end{verbatim}
The dash before the \code{include} tells \code{make} to ignore
errors so that \code{make depend} will succeed before the file
exists. The \code{MakeDepend} file will contain the include and
module dependencies for each source file, such as:
\begin{verbatim}
Build/mod_diags.o: tile.h cppdefs.h globaldefs.h shoreface.h
Build/mod_diags.f90: tile.h cppdefs.h globaldefs.h shoreface.h
Build/mod_diags.o: Build/mod_kinds.o Build/mod_param.o Build/mod_diags.f90
\end{verbatim}
Note that the \code{.h} files are included by \code{cpp}, so that
both the \code{.f90} and \code{.o} files become out of date when an
include file is modified. Without the module dependencies,
\code{make} would try to build the sources in the wrong order and
the compiler would fail with a complaint about not finding
\code{mod\_param.mod}, for instance.
\end{enumerate}

\subsection{Final warnings}

The cost of this nifty \code{make} stuff is:
\begin{enumerate}
\item We're a little closer to the \code{gnu make} bugs here, and we
need a newer version of \code{gnu make} than before (version 3.81,
3.80 if you're lucky).
Hence this stuff at the top of the \code{makefile}:
\begin{verbatim}
NEED_VERSION := 3.80 3.81 3.82
$(if $(filter $(MAKE_VERSION),$(NEED_VERSION)),,        \
 $(error This makefile requires one of GNU make version $(NEED_VERSION).))
\end{verbatim}

\item The \code{Makefile} dependencies get just a little trickier every
change we make. Note that \code{F90} has potentially both \code{include}
and \code{module} use dependencies. The book example uses the C compiler
to produce its own dependencies for each source file into a corresponding
\code{.d} file to be included by \code{make}. Our Fortran compilers are
not so smart. For these hairy compiles, it's critical to have accurate
dependency information unless we're willing to \code{make clean} between
compiles.
\end{enumerate}
