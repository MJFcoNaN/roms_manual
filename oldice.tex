%The viscous-plastic terms (\ref{fx1}) and (\ref{fy1})
%are found by taking the divergence of the stress tensor:
%\begin{equation}
%  ({\cal F}_x, {\cal F}_y) = \nabla \cdot \mbox{\boldmath $\sigma$}
%\end{equation}
%with the result that
%\begin{equation}
%   {\cal F}_x = {\partial \over \partial x} \left[ (\eta + \zeta)
%   {\partial u \over \partial x} + (\zeta - \eta)
%   {\partial v \over \partial y} - P/2 \right] +
%   {\partial \over \partial y} \left[ \eta \left( 
%   {\partial u \over \partial y} + {\partial v \over \partial x}
%   \right) \right]
%\label{fx1}
%\end{equation}
%\begin{equation}
%   {\cal F}_y = {\partial \over \partial y} \left[ (\eta + \zeta)
%   {\partial v \over \partial y} + (\zeta - \eta)
%   {\partial u \over \partial x} - P/2 \right] +
%   {\partial \over \partial x} \left[ \eta \left( 
%   {\partial u \over \partial y} + {\partial v \over \partial x}
%   \right) \right]
%\label{fy1}
%\end{equation}

%The ``pressure gradient'' term is also modeled as a term in the
%internal ice stress.  This term represents the resistance which ice
%has to being compressed (ice strength) and is a function of ice
%thickness and concentration:
%\begin{equation}
%  P = P^* A h_i \exp[ -C (1-A)] H(-\nabla \cdot \vec{v})~.
%\label{press}
%\end{equation}
%The Heaviside function guarantees that the ice has no strength when the
%flow is divergent \citep{Gray95}.

\subsection{Horizontal curvilinear coordinates}

Applying the curvilinear transformation used in \S\ref{Ocurve} and
described in Appendix \ref{Curve}, we use a transformation to an
orthogonal curvilinear coordinate system.  Denoting the velocity
components in the new coordinate system by
\begin{equation}
   \vec{v} \cdot \hat{\xi} = u
\end{equation}
and
\begin{equation}
   \vec{v} \cdot \hat{\eta} = v
\end{equation}
the equations of motion (\ref{ist1}), (\ref{ist2}),
(\ref{ist3a})--(\ref{ist3b}) can be re-written
\citep[see, e.g.,][]{AL}:
{\samepage
%\[
 %  {\partial u \over \partial t}
 %  + m u {\partial u \over \partial \xi}
 %  + n v {\partial u \over \partial \eta}
 %  = \left\{f + mn \left[ v \frac{\partial}{\partial \xi}
 %  \!\! \left( \frac{1}{n} \right) - u \frac{\partial}{\partial \eta}
 %  \!\! \left( \frac{1}{m} \right) \right] \right\} v
%\]
\begin{equation}
   {\partial u \over \partial t} = fv
   - gm {\partial \zeta_w \over \partial \xi} +
   {1 \over M} \left( \tau_a^{\xi} + \tau_w^{\xi} + {\cal F}_{\xi}
   \right)
\label{ist13}
\end{equation}
}
\vspace{.2cm}
{\samepage
%\[
%   {\partial v \over \partial t} +
%   m u {\partial v \over \partial \xi} +
%   n v {\partial v \over \partial \eta} =
%   - \left\{f + mn \left[ v \frac{\partial}{\partial \xi}
%   \!\! \left( \frac{1}{n} \right) - u \frac{\partial}{\partial \eta}
%   \!\! \left( \frac{1}{m} \right) \right] \right\} u
%\]
\begin{equation}
   {\partial v \over \partial t} = - fu
   - gn {\partial \zeta_w \over \partial \eta} +
   {1 \over M} \left( \tau_a^{\eta} + \tau_w^{\eta} + {\cal F}_{\eta}
   \right)
\label{ist14}
\end{equation}
}
\begin{equation}
   {\partial A h_i \over \partial t} =
   - mn \left[ {\partial \over \partial \xi} \!\!
   \left( {A h_i u \over n}
   \right ) + {\partial \over \partial \eta} \!\!
   \left( {A h_i v \over m}
   \right) \right] + S_h
\label{ist15}
\end{equation}
\begin{equation}
   {\partial A h_s \over \partial t} =
   - mn \left[ {\partial \over \partial \xi} \!\!
   \left( {A h_s u \over n}
   \right ) + {\partial \over \partial \eta} \!\!
   \left( {A h_s v \over m}
   \right) \right] + S_s
\label{ist15a}
\end{equation}
\begin{equation}
   {\partial A \over \partial t} =
   - mn \left[ {\partial \over \partial \xi} \!\! \left( {Au \over n}
   \right ) + {\partial \over \partial \eta} \!\! \left( {Av \over m}
   \right) \right] + S_A
\label{ist16}
\end{equation}
\vspace{.2cm}
$S_h$, $S_s$ and $S_A$ remain unchanged.

The viscous-plastic terms can be derived from equation
(\ref{stress1}).  In curvilinear coordinates the strain rate tensor
can be written as:
\begin{equation}
   \epsilon_{11} = m {\partial u \over \partial \xi} +
    vmn {\partial \over \partial \eta} \!\! \left( {1 \over m} \right)
\end{equation}
\begin{equation}
   \epsilon_{12} = e_{21} = {1 \over 2} \left[
   {m \over n} {\partial \left( nv \right) \over \partial \xi}
   + {n \over m} {\partial \left( mu \right) \over \partial \eta}
   \right]
\end{equation}
\begin{equation}
   \epsilon_{22} = n {\partial v \over \partial \eta} +
    umn {\partial \over \partial \xi} \!\! \left( {1 \over n} \right)
\end{equation}
In curvilinear coordinates the divergence of a symmetric tensor {\bf T}
is:
\[
    \nabla \cdot \mbox{\bf T} =
    \hat{\xi} \left[ m
    {\partial \mbox{\bf T}_{11} \over \partial \xi}
    + n {\partial \mbox{\bf T}_{12} \over \partial \eta}
    + \mbox{\bf T}_{11} mn {\partial \over \partial \xi} \!\!
    \left( {1 \over n} \right) \right. \mbox{\hspace{2 cm}}
\]
\[
    \mbox{\hspace{2 cm}} \left. + 2 \mbox{\bf T}_{12} mn
    {\partial \over \partial \eta} \!\! \left( {1 \over m} \right)
    - \mbox{\bf T}_{22} mn
    {\partial \over \partial \xi} \!\! \left( {1 \over n} \right)
    \right]
\]
\[
    + \hat{\eta} \left[ m
    {\partial \mbox{\bf T}_{12} \over \partial \xi}
    + n {\partial \mbox{\bf T}_{22} \over \partial \eta}
    - \mbox{\bf T}_{11} mn {\partial \over \partial \eta} \!\!
    \left( {1 \over m} \right) \right. \mbox{\hspace{2 cm}}
\]
\begin{equation}
    \mbox{\hspace{2 cm}} \left. + 2 \mbox{\bf T}_{12} mn
    {\partial \over \partial \xi} \!\! \left( {1 \over n} \right)
    + \mbox{\bf T}_{22} mn
    {\partial \over \partial \eta} \!\! \left( {1 \over m} \right)
    \right]
\end{equation}
A more general expression for derivatives of tensors is given by
\citet{Aris} in terms of Christoffel symbols.

The viscous-plastic terms become:
\[
   {\cal F}_{\xi} = m {\partial \over \partial \xi} \left[
   (\zeta - \eta ) mn {\partial \over \partial \xi} \!\! \left(
   {u \over n} \right) \right] +
   m {\partial \over \partial \xi} \left[
   (\zeta - \eta ) mn {\partial \over \partial \eta} \!\! \left(
   {v \over m} \right) \right] - {m \over 2}
   {\partial P \over \partial \xi}
\]
\[
   + 2 m {\partial \over \partial \xi} \left[ \eta m
   {\partial u \over \partial \xi} + \eta v mn 
   {\partial \over \partial \eta} \!\! \left( {1 \over m} \right)
   \right]
   + n {\partial \over \partial \eta} \left[ \eta {m \over n}
   {\partial (nv) \over \partial \xi} + \eta {n \over m}
   {\partial (mu) \over \partial \eta} \right]
\]
\[
   + 2 \eta m^2 n {\partial u \over \partial \xi}
   {\partial \over \partial \xi} \!\! \left( {1 \over n} \right)
   + 2 \eta v m^2 n^2
   {\partial \over \partial \eta} \!\! \left( {1 \over m} \right)
   {\partial \over \partial \xi} \!\! \left( {1 \over n} \right)
   + 2 \eta m^2 {\partial (nv) \over \partial \xi}
   {\partial \over \partial \eta} \!\! \left( {1 \over m} \right)
\]
\begin{equation}
   + 2 \eta n^2 {\partial (mu) \over \partial \eta}
   {\partial \over \partial \eta} \!\! \left( {1 \over m} \right)
   - 2 \eta m n^2 {\partial v \over \partial \eta}
   {\partial \over \partial \xi} \!\! \left( {1 \over n} \right)
   - 2 \eta u m^2 n^2 \left[
   {\partial \over \partial \xi} \!\! \left( {1 \over n} \right)
   \right]^2
\end{equation}
\vspace{2 mm}
\[
   {\cal F}_{\eta} = n {\partial \over \partial \eta} \left[
   (\zeta - \eta ) mn {\partial \over \partial \xi} \!\! \left(
   {u \over n} \right) \right] +
   n {\partial \over \partial \eta} \left[
   (\zeta - \eta ) mn {\partial \over \partial \eta} \!\! \left(
   {v \over m} \right) \right] - {n \over 2}
   {\partial P \over \partial \eta}
\]
\[
   + m {\partial \over \partial \xi} \left[ \eta {m \over n}
   {\partial (nv) \over \partial \xi} + \eta {n \over m}
   {\partial (mu) \over \partial \eta} \right]
   + 2n {\partial \over \partial \eta} \left[ \eta n
   {\partial v \over \partial \eta} + \eta u mn 
   {\partial \over \partial \xi} \!\! \left( {1 \over n} \right)
   \right]
\]
\[
   - 2 \eta m^2 n {\partial u \over \partial \xi}
   {\partial \over \partial \eta} \!\! \left( {1 \over m} \right)
   - 2 \eta v m^2 n^2 \left[
   {\partial \over \partial \eta} \!\! \left( {1 \over m} \right)
   \right]^2
   + 2 \eta m^2 {\partial (nv) \over \partial \xi}
   {\partial \over \partial \xi} \!\! \left( {1 \over n} \right)
\]
\begin{equation}
   + 2 \eta n^2 {\partial (mu) \over \partial \eta}
   {\partial \over \partial \xi} \!\! \left( {1 \over n} \right)
   + 2 \eta m n^2 {\partial v \over \partial \eta}
   {\partial \over \partial \eta} \!\! \left( {1 \over m} \right)
   + 2 \eta u m^2 n^2
   {\partial \over \partial \eta} \!\! \left( {1 \over m} \right)
   {\partial \over \partial \xi} \!\! \left( {1 \over n} \right)
\end{equation}

\subsection{Numerical Scheme}

A key feature of the numerical scheme is a staggered grid, known as
the ``Arakawa C grid'', where the velocity is defined on the sides of a
grid box and variables such as thickness and viscosity are defined
at the center of the grid box, as shown in Fig.\ \ref{fcgr}.

Because of the strong ice interaction (which in this model is
dissipative in nature) the momentum equations are essentially
parabolic in form and hence have few numerical instability problems
over long-term integrations.  (While there are few numerical
problems, it should be emphasized that the dissipative ice
interaction terms are highly nonlinear and can lead to unstable flow
fields in the absence of water drag.  Such a feature is a physical
characteristic of plastic flow and not a numerical artifact).
However, in principle it is possible for the ice interaction to be
very small even though there may be a finite ice mass.  Under this
situation, the momentum advection terms could cause nonlinear
instabilities.  To ensure against such situations, Hibler put a lower
limit on the bulk viscosity parameter
(and hence indirectly shear viscosity as well). It is never
allowed to drop below $1.0 \times 10^7$ $kg/s$, a value which
negligibly modifies the ice drift.
% We have not found this to be
% necessary and therefore do not limit the viscosities.

Nonlinear instabilities over long-term integrations can also arise from
the nonlinear advection terms in the thickness continuity equations.
To avoid this problem, the advection terms in equations
(\ref{ist3a})--(\ref{ist3b}) can be computed with a choice of either
MPDATA \citep{Smolark90} or third-order upwind
\citep{Leonard79}. See \S\ref{Advect} for a description of
these schemes. It may be possible to omit the diffusivity from these
equations if the forcing fields are sufficiently smooth.

\subsection{Horizontal boundary conditions}

As mentioned above, initial conditions at all points and ice
velocities at the boundaries are thereafter required to initiate the
integration of the system of equations forward in time.  The most
natural boundary condition is to take the ice velocity to be zero on
the boundaries.  This can be done either at a land boundary or at an
ocean location where there is no ice.  Note that the boundary
condition does not affect the ice motion in such circumstances since
in the absence of ice the strength is zero.  More generally, as long
as the ocean boundaries are removed from the ice edge, the coupled
nature of the model will cause a natural ice edge boundary condition
to be created.  However, it is also possible to form an ``open''
boundary condition by setting the strength equal to zero near a
boundary.  These gridboxes are called ``outflow cells'' in
\citet{Hibler79}.  In the Arctic simulations, these outflow cells
are used at the open edge near Greenland.

The boundary conditions on the momentum equations are to set $u$ and $v$
to zero at all boundaries, including islands.  This is accomplished
by the elliptic solver during the implicit timestep.

The ice and snow thickness and ice concentration equations have no-flux
boundary conditions imposed along the mask boundaries.  The outflow
cells contain a radiation condition if the velocity is outward and no
change if the velocity is inward.

The primary characteristic of the outflow cells is that the ice
strength goes to zero there.  The values of $P$, $\zeta$, and $\eta$
are all set to zero in outflow cells.
