\setcounter{page}{0}
This document was prepared with \LaTeX\, xfig, and inkscape.

\mbox{  }
\begin{center}
\bf \LARGE Acknowledgments
\end{center}

The ROMS model is descended from the SPEM and SCRUM models, but has
been entirely rewritten by Sasha Shchepetkin, Hernan Arango and John
Warner, with many, many other contributors. I am indebted to every one
of them for their hard work.

Bill Hibler first came up with the viscous-plastic rheology we are
using. Paul Budgell has rewritten the dynamic sea-ice model, improving
the solution procedure and making the water-stress term implicit in time,
then changing it again to use the elastic-viscous-plastic rheology of
Hunke and Dukowicz. I am very grateful that he is allowing us to use
his version of the code. The sea-ice thermodynamics is derived from
Sirpa H\"akkinen's implementation of the Mellor-Kantha scheme. She was
kind enough to allow Paul and I to start with her code.

Thanks to the internet community for providing great tools like Perl,
patch, cpp, svn, and gmake to aid in software development (and to make
it more fun).

This work was supported in part by a grant of HPC resources from the
Arctic Region Supercomputing Center and the DoD High Performance
Computing Modernization Program.

Development and testing of the ROMS model has been funded by many,
including the USGS Coastal and Marine Program, the Office of Naval
Research, the National Ocean Partnership Program...

\vspace{\fill}
UNIX is a registered trademark of the Open Group.

Cygwin is a registered trademark of Red Hat, Inc.


\vfil\break
\begin{abstract}
The Regional Ocean Modeling System (ROMS), authored by many, most
notably Sasha Shchepetkin, is one approach to regional and basin-scale ocean
modeling. This user's manual for ROMS describes the model equations
and algorithms, as well as additional user configurations necessary
for specific applications. ROMS itself has now branched out as
well - the version described here is that available through the
myroms.org svn site with modifications to include sea ice and other
minor changes.

\end{abstract}
