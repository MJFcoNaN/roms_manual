\section{The \code{patch} program}
\label{Patch}
We sometimes discover things in SCRUM which we would like to modify,
either to fix bugs, or to add new features.
Hernan Arango keeps track of these changes
and periodically sends patches to the
list of known SCRUM users so that they can update their versions.  By
sending out these changes rather than the whole updated model, people
can acquire bug fixes and still retain the changes they have made to
SCRUM for their own applications.

Larry Wall has written a program to take the output of \code{diff} and
automatically apply it to the old version of a file to create the new
version.  This program is called \code{patch} and is available from all
the \code{gnu} archive sites.  If the output of \code{diff} has been
saved in the file \code{scrum.patch.20} then \code{patch} would be used
as follows:
\begin{verbatim}
        patch < scrum.patch.20
\end{verbatim}
As \code{patch} updates the files, it leaves the original of
\code{file} in \code{file.orig}.  If it gets confused for some reason
(if you modified the lines of code \code{patch} wants to change)
it will create a \code{file.rej} file.  I
often check to see if any \code{.rej} files get created---these can be
used to patch \code{file} by hand and can then be deleted.
