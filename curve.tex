\section{Horizontal curvilinear coordinates}
\label{Curve}
The requirement for a boundary-following coordinate system and for a
laterally variable grid resolution can both be met (for suitably
smooth domains) by introducing an appropriate orthogonal coordinate
transformation in the horizontal.  Let the new coordinates be
$\xi(x,y)$ and $\eta(x,y)$ where the relationship of horizontal arc
length to the differential distance is given by:
\begin{equation}
   (ds)_{\xi} = \left( {1 \over m} \right) d \xi
\end{equation}
\begin{equation}
   (ds)_{\eta} = \left( {1 \over n} \right) d \eta
\end{equation}
Here, $m(\xi,\eta)$ and $n(\xi,\eta)$ are the scale factors which
relate the differential distances $(\Delta \xi,\Delta \eta)$ to the
actual (physical) arc lengths.

It is helpful to write the equations in vector notation and to use
the formulas for div, grad, and curl in curvilinear coordinates (see
Batchelor, Appendix 2, \cite{Batchelor}):
\begin{equation}
   \nabla \phi = \hat{\xi} m {\partial \phi \over \partial \xi} +
                 \hat{\eta} n {\partial \phi \over \partial \eta}
\end{equation}
\begin{equation}
   \nabla \cdot \vec{a} = mn \left[
   {\partial \over \partial \xi} \!\! \left( {a \over n} \right) +
   {\partial \over \partial \eta} \!\! \left( {b \over m} \right)
   \right]
\end{equation}
\begin{equation}
   \nabla \times \vec{a} = mn \left| \begin{array}{ccc}
   \vspace{1 mm}
   {\hat{\xi}_1 \over m} & {\hat{\xi}_2 \over n} & \hat{k} \\
   \vspace{1 mm}
   {\partial \over \partial \xi} &
   {\partial \over \partial \eta} &
   {\partial \over \partial z} \\
   {a \over m} & {b \over n} & c
   \end{array} \right|
\end{equation}
\begin{equation}
   \nabla^2 \phi = \nabla \cdot \nabla \phi = mn \left[ 
   {\partial \over \partial \xi} \!\! \left( {m \over n} 
   {\partial \phi \over \partial \xi} \right) +
   {\partial \over \partial \eta} \!\! \left( {n \over m} 
   {\partial \phi \over \partial \eta} \right) \right]
\end{equation}
where $\phi$ is a scalar and $\vec{a}$ is a vector with components
$a$, $b$, and $c$.
