\chapter{Model Formulation}
\label{Phys}
\section {Equations of motion}
The equations of motion for the ice model in Cartesian coordinates can
be written \citep[from][]{Hibler79}:
\begin{equation}
  M {\partial u \over \partial t} + M \vec{v} \cdot \nabla u = Mfv
  - Mg {\partial H \over \partial x} + \tau_a^x + \tau_w^x + {\cal F}_x
\label{st1}
\end{equation}
\vspace{.2cm}
\begin{equation}
  M {\partial v \over \partial t} + M \vec{v} \cdot \nabla v = - Mfu
  - Mg {\partial H \over \partial y} + \tau_a^y + \tau_w^y + {\cal F}_y
\label{st2}
\end{equation}
\vspace{.2cm}
\begin{equation}
  \sigma_{ij} = 2 \eta \epsilon_{ij} + (\zeta - \eta) \epsilon_{kk}
  \delta_{ij} - {P \over 2} \delta_{ij}
\label{stress1}
\end{equation}
\vspace{.2cm}
\begin{equation}
   {\cal F}_x = {\partial \over \partial x} \left[ (\eta + \zeta)
   {\partial u \over \partial x} + (\zeta - \eta)
   {\partial v \over \partial y} - P/2 \right] +
   {\partial \over \partial y} \left[ \eta \left( 
   {\partial u \over \partial y} + {\partial v \over \partial x}
   \right) \right]
\label{fx1}
\end{equation}
\vspace{.2cm}
\begin{equation}
   {\cal F}_y = {\partial \over \partial y} \left[ (\eta + \zeta)
   {\partial v \over \partial y} + (\zeta - \eta)
   {\partial u \over \partial x} - P/2 \right] +
   {\partial \over \partial x} \left[ \eta \left( 
   {\partial u \over \partial y} + {\partial v \over \partial x}
   \right) \right]
\label{fy1}
\end{equation}
\vspace{.2cm}
\begin{equation}
  {\partial h \over \partial t} =
  - {\partial (uh) \over \partial x} - {\partial (vh) \over \partial y}
  + S_h + {\cal D}
\label{st3a}
\end{equation}
\vspace{.2cm}
\begin{equation}
  {\partial A \over \partial t} =
  - {\partial (uA) \over \partial x} - {\partial (vA) \over \partial y}
  + S_A + {\cal D}
\label{st3b}
\end{equation}
\vspace{.2cm}
\begin{equation}
   S_h = A \, G \! \left( {h \over A} \right) + (1-A)G(0)
\end{equation}
{\samepage
\[
   S_A = \left\{ \begin{array}{cl}
       {G(0)(1-A) \over h_o},  &  \mbox{if \quad} G(0) > 0, \\
       0,               &  \mbox{if \quad} G(0) < 0,
       \end{array} \right.
\]
\begin{equation}
   + \left\{ \begin{array}{cl}
       0,               &  \mbox{if \quad} S_h > 0, \\
       {S_h A \over 2h}  &  \mbox{if \quad} S_h < 0,
       \end{array} \right.
\end{equation}
}
where, in standard notation:
\begin{center}
  ($u,v$) = the ($x,y$) components of vector velocity $\vec{v}$ \\
\vspace{1mm}
  $h(x,y,t)$ = ice thickness \\
\vspace{1mm}
  $h_o$ = ice cutoff thickness \\
\vspace{1mm}
  $A(x,y,t)$ = ice concentration \\
\vspace{1mm}
  $M(x,y,t)$ = ice mass (density times thickness) \\
\vspace{1mm}
  $G(h)$ = growth rate of ice of thickness $h$ \\
\vspace{1mm}
  $H(x,y,t)$ = height of the ocean surface \\
\vspace{1mm}
  $\zeta(x,y,t)$ = nonlinear bulk viscosity \\
\vspace{1mm}
  $\eta(x,y,t)$ = nonlinear shear viscosity \\
\vspace{1mm}
  $f(x,y)$ = Coriolis parameter \\
\vspace{1mm}
  $g$ = acceleration of gravity \\
\vspace{1mm}
  $\sigma_{ij}(x,y,t)$ = stress tensor \\
\vspace{1mm}
  $\epsilon_{ij}(x,y,t)$ = strain rate tensor \\
\vspace{1mm}
  $P(x,y,t)$ = ice pressure or strength \\
\vspace{1mm}
  ($\tau_a, \tau_w$) = air and water stresses \\
\vspace{1mm}
  (${\cal F}_x, {\cal F}_y$) = viscous-plastic terms \\
\vspace{1mm}
  ($S_h, S_A$) = thermodynamic terms
\end{center}
and
\begin{center}
  (${\cal D}_h, {\cal D}_A$) = diffusive terms.
\end{center}
Equations (\ref{st1}) and (\ref{st2}) express the momentum balance in
the $x$ and $y$ directions, respectively.  The time evolution of the
thickness and concentration fields, $h(x,y,t)$ and $A(x,y,t)$, are
governed by the advective-diffusive equations (\ref{st3a}) and
(\ref{st3b}).  The viscous-plastic terms (\ref{fx1}) and (\ref{fy1})
are found by taking the divergence of the stress tensor
(\ref{stress1}).  For the moment, the effects of dissipation will be
represented by the schematic terms ${\cal D}$, which are the sum of a
Laplacian and biharmonic diffusion.

%\section{Horizontal boundary conditions}
%As distributed, the model configuration is that of a periodic channel.
%If the variable \code{perchan} is set to \code{false} then the boundary
%conditions are that of a closed basin.  For any other configuration it
%is the responsibility of the user to provide the appropriate boundary
%conditions on $u,v,T,S,$ and $\psi$, where $\psi$ is the transport
%streamfunction.  At every time step the subroutines \code{bcs} and
%\code{bcpsi} are called to fill in the necessary boundary values.

%If the biharmonic friction is used, a
%higher order boundary condition must also be provided.  The model
%currently has this built into the code where the biharmonic terms are
%calculated.  The high order boundary conditions used are
%${\partial \over \partial x} \left( {h\nu \over mn} {\partial
%^2 u \over \partial x^2} \right) = 0$
%on the eastern and western boundaries and
%${\partial \over \partial y} \left( {h\nu \over mn} {\partial
%^2 u \over \partial y^2} \right) = 0$ on the northern
%and southern boundaries.  The boundary conditions for $v,T,$ and $S$
%are similar.  These boundary conditions were chosen because they
%preserve the property of no gain or loss of momentum, temperature,
%or salt across a closed boundary.

\section{Horizontal curvilinear coordinates}
In many applications of interest (e.g., flow adjacent to a coastal
boundary), the fluid may be confined horizontally within an
irregular region.  In such problems, a horizontal coordinate system
which conforms to the irregular lateral boundaries is advantageous.
It is often also true in many geophysical problems that the
simulated flow fields have regions of enhanced structure (e.g.,
boundary currents or fronts) which occupy a relatively small
fraction of the physical /computational domain.  In these problems,
added efficiency can be gained by placing more (computational)
resolution in such regions.

The requirement for a boundary-following coordinate system and for a
laterally variable grid resolution can both be met (for suitably
smooth domains) by introducing an appropriate orthogonal coordinate
transformation in the horizontal.  Let the new coordinates be $\xi_1
(x,y)$ and $\xi_2(x,y)$ where the relationship of horizontal arc
length to the differential distance is given by:
\begin{equation}
   (ds)_{\xi_1} = \left( {1 \over m} \right) d \xi_1
\end{equation}
\begin{equation}
   (ds)_{\xi_2} = \left( {1 \over n} \right) d \xi_2
\end{equation}
Here, $m(\xi_1,\xi_2)$ and $n(\xi_1,\xi_2)$ are the scale factors which
relate the differential distances $(\Delta \xi_1,\Delta \xi_2)$ to the
actual (physical) arc lengths.

It is helpful to write the equations in vector notation and to use
the formulas for div, grad, and curl in curvilinear coordinates
\citep[see][Appendix 2]{Batchelor}:
\begin{equation}
   \nabla \phi = \hat{\xi_1} m {\partial \phi \over \partial \xi_1} +
                 \hat{\xi_2} n {\partial \phi \over \partial \xi_2}
\end{equation}
\begin{equation}
   \nabla \cdot \vec{a} = mn \left[
   {\partial \over \partial \xi_1} \!\! \left( {a \over n} \right) +
   {\partial \over \partial \xi_2} \!\! \left( {b \over m} \right)
   \right]
\end{equation}
\begin{equation}
   \nabla \times \vec{a} = mn \left| \begin{array}{ccc}
   \vspace{1 mm}
   {\hat{\xi}_1 \over m} & {\hat{\xi}_2 \over n} & \hat{k} \\
   \vspace{1 mm}
   {\partial \over \partial \xi_1} &
   {\partial \over \partial \xi_2} &
   {\partial \over \partial \xi_3} \\
   {a \over m} & {b \over n} & c
   \end{array} \right|
\end{equation}
\begin{equation}
   \nabla^2 \phi = \nabla \cdot \nabla \phi = mn \left[ 
   {\partial \over \partial \xi_1} \!\! \left( {m \over n} 
   {\partial \phi \over \partial \xi_1} \right) +
   {\partial \over \partial \xi_2} \!\! \left( {n \over m} 
   {\partial \phi \over \partial \xi_2} \right) \right]
\end{equation}
where $\phi$ is a scalar and $\vec{a}$ is a vector with components
$a$, $b$, and $c$.

Denoting the velocity components in the new coordinate system by
\begin{equation}
   \vec{v} \cdot \hat{\xi_1} = u
\end{equation}
and
\begin{equation}
   \vec{v} \cdot \hat{\xi_2} = v
\end{equation}
the equations of motion (\ref{st1})-(\ref{st3b}) can be re-written
\citep[see, e.g.,][]{AL}:
{\samepage
\[
   {\partial u \over \partial t} +
   m u {\partial u \over \partial \xi_1} +
   n v {\partial u \over \partial \xi_2} =
   \left\{f + mn \left[ v \frac{\partial}{\partial \xi_1}
   \!\! \left( \frac{1}{n} \right) - u \frac{\partial}{\partial \xi_2}
   \!\! \left( \frac{1}{m} \right) \right] \right\} v
\]
\begin{equation}
   - gm {\partial H \over \partial \xi_1} +
   {1 \over M} \left( \tau_a^{\xi_1} + \tau_w^{\xi_1} + {\cal F}_{\xi_1}
   \right)
\label{st13}
\end{equation}
}
\vspace{.2cm}
{\samepage
\[
   {\partial v \over \partial t} +
   m u {\partial v \over \partial \xi_1} +
   n v {\partial v \over \partial \xi_2} =
   - \left\{f + mn \left[ v \frac{\partial}{\partial \xi_1}
   \!\! \left( \frac{1}{n} \right) - u \frac{\partial}{\partial \xi_2}
   \!\! \left( \frac{1}{m} \right) \right] \right\} u
\]
\begin{equation}
   - gn {\partial H \over \partial \xi_2} +
   {1 \over M} \left( \tau_a^{\xi_2} + \tau_w^{\xi_2} + {\cal F}_{\xi_2}
   \right)
\label{st14}
\end{equation}
}
\vspace{.2cm}
\begin{equation}
   {\partial h \over \partial t} =
   - mn \left[ {\partial \over \partial \xi_1} \!\! \left( {hu \over n}
   \right ) + {\partial \over \partial \xi_2} \!\! \left( {hv \over m}
   \right) \right] + S_h + {\cal D}_h
\label{st15}
\end{equation}
\vspace{.2cm}
\begin{equation}
   {\partial A \over \partial t} =
   - mn \left[ {\partial \over \partial \xi_1} \!\! \left( {Au \over n}
   \right ) + {\partial \over \partial \xi_2} \!\! \left( {Av \over m}
   \right) \right] + S_A + {\cal D}_A
\label{st16}
\end{equation}
\vspace{.2cm}
$S_h$ and $S_A$ remain unchanged.

The viscous-plastic terms can be derived from equation
(\ref{stress1}).  In curvilinear coordinates the strain rate tensor
can be writen as:
\begin{equation}
   \epsilon_{11} = m {\partial u \over \partial \xi_1} +
    vmn {\partial \over \partial \xi_2} \!\! \left( {1 \over m} \right)
\end{equation}
\begin{equation}
   \epsilon_{12} = e_{21} = {1 \over 2} \left[
   {m \over n} {\partial \left( nv \right) \over \partial \xi_1}
   + {n \over m} {\partial \left( mu \right) \over \partial \xi_2}
   \right]
\end{equation}
\begin{equation}
   \epsilon_{22} = n {\partial v \over \partial \xi_2} +
    umn {\partial \over \partial \xi_1} \!\! \left( {1 \over n} \right)
\end{equation}
In curvilinear coordinates the divergence of symmetric tensor {\bf T}
is:
\[
    \nabla \cdot \mbox{\bf T} =
    \hat{\xi_1} \left[ m
    {\partial \mbox{\bf T}_{11} \over \partial \xi_1}
    + n {\partial \mbox{\bf T}_{12} \over \partial \xi_2}
    + \mbox{\bf T}_{11} mn {\partial \over \partial \xi_1} \!\!
    \left( {1 \over n} \right) \right. \mbox{\hspace{2 cm}}
\]
\[
    \mbox{\hspace{2 cm}} \left. + 2 \mbox{\bf T}_{12} mn
    {\partial \over \partial \xi_2} \!\! \left( {1 \over m} \right)
    - \mbox{\bf T}_{22} mn
    {\partial \over \partial \xi_1} \!\! \left( {1 \over n} \right)
    \right]
\]
\[
    + \hat{\xi_2} \left[ m
    {\partial \mbox{\bf T}_{12} \over \partial \xi_1}
    + n {\partial \mbox{\bf T}_{22} \over \partial \xi_2}
    - \mbox{\bf T}_{11} mn {\partial \over \partial \xi_2} \!\!
    \left( {1 \over m} \right) \right. \mbox{\hspace{2 cm}}
\]
\begin{equation}
    \mbox{\hspace{2 cm}} \left. + 2 \mbox{\bf T}_{12} mn
    {\partial \over \partial \xi_1} \!\! \left( {1 \over n} \right)
    + \mbox{\bf T}_{22} mn
    {\partial \over \partial \xi_2} \!\! \left( {1 \over m} \right)
    \right]
\end{equation}
A more general expression for derivatives of tensors is given by
\citet{Aris} in terms of Christoffel symbols.

The viscous-plastic terms become:
\[
   {\cal F}_{\xi_1} = m {\partial \over \partial \xi_1} \left[
   (\zeta - \eta ) mn {\partial \over \partial \xi_1} \!\! \left(
   {u \over n} \right) \right] +
   m {\partial \over \partial \xi_1} \left[
   (\zeta - \eta ) mn {\partial \over \partial \xi_2} \!\! \left(
   {v \over m} \right) \right] - {m \over 2}
   {\partial P \over \partial \xi_1}
\]
\[
   + 2 m {\partial \over \partial \xi_1} \left[ \eta m
   {\partial u \over \partial \xi_1} + \eta v mn 
   {\partial \over \partial \xi_2} \!\! \left( {1 \over m} \right)
   \right]
   + n {\partial \over \partial \xi_2} \left[ \eta {m \over n}
   {\partial (nv) \over \partial \xi_1} + \eta {n \over m}
   {\partial (mu) \over \partial \xi_2} \right]
\]
\[
   + 2 \eta m^2 n {\partial u \over \partial \xi_1}
   {\partial \over \partial \xi_1} \!\! \left( {1 \over n} \right)
   + 2 \eta v m^2 n^2
   {\partial \over \partial \xi_2} \!\! \left( {1 \over m} \right)
   {\partial \over \partial \xi_1} \!\! \left( {1 \over n} \right)
   + 2 \eta m^2 {\partial (nv) \over \partial \xi_1}
   {\partial \over \partial \xi_2} \!\! \left( {1 \over m} \right)
\]
\begin{equation}
   + 2 \eta n^2 {\partial (mu) \over \partial \xi_2}
   {\partial \over \partial \xi_2} \!\! \left( {1 \over m} \right)
   - 2 \eta m n^2 {\partial v \over \partial \xi_2}
   {\partial \over \partial \xi_1} \!\! \left( {1 \over n} \right)
   - 2 \eta u m^2 n^2 \left[
   {\partial \over \partial \xi_1} \!\! \left( {1 \over n} \right)
   \right]^2
\end{equation}
\vspace{2 mm}
\[
   {\cal F}_{\xi_2} = n {\partial \over \partial \xi_2} \left[
   (\zeta - \eta ) mn {\partial \over \partial \xi_1} \!\! \left(
   {u \over n} \right) \right] +
   n {\partial \over \partial \xi_2} \left[
   (\zeta - \eta ) mn {\partial \over \partial \xi_2} \!\! \left(
   {v \over m} \right) \right] - {n \over 2}
   {\partial P \over \partial \xi_2}
\]
\[
   + m {\partial \over \partial \xi_1} \left[ \eta {m \over n}
   {\partial (nv) \over \partial \xi_1} + \eta {n \over m}
   {\partial (mu) \over \partial \xi_2} \right]
   + 2n {\partial \over \partial \xi_2} \left[ \eta n
   {\partial v \over \partial \xi_2} + \eta u mn 
   {\partial \over \partial \xi_1} \!\! \left( {1 \over n} \right)
   \right]
\]
\[
   - 2 \eta m^2 n {\partial u \over \partial \xi_1}
   {\partial \over \partial \xi_2} \!\! \left( {1 \over m} \right)
   - 2 \eta v m^2 n^2 \left[
   {\partial \over \partial \xi_2} \!\! \left( {1 \over m} \right)
   \right]^2
   + 2 \eta m^2 {\partial (nv) \over \partial \xi_1}
   {\partial \over \partial \xi_1} \!\! \left( {1 \over n} \right)
\]
\begin{equation}
   + 2 \eta n^2 {\partial (mu) \over \partial \xi_2}
   {\partial \over \partial \xi_1} \!\! \left( {1 \over n} \right)
   + 2 \eta m n^2 {\partial v \over \partial \xi_2}
   {\partial \over \partial \xi_2} \!\! \left( {1 \over m} \right)
   + 2 \eta u m^2 n^2
   {\partial \over \partial \xi_2} \!\! \left( {1 \over m} \right)
   {\partial \over \partial \xi_1} \!\! \left( {1 \over n} \right)
\end{equation}
