\subsubsection{Smolarkiewicz}
\label{Smolark}
Smolarkiewicz has written a series of papers (\cite{Smolark83},
\cite{Smolark84}, \cite{Smolark86} and \cite{Smolark90}) on an
advection scheme known as MPDATA (Multidimensional
Positive Definite Advection Transport Algorithm).
MPDATA is meant to be applied to positive-definite tracer fields
only. It uses a first-order upwind scheme to obtain a first
estimate of the updated field, followed by one or more
``antidiffusion'' passes to obtain the final updated field.
Note that we add an offset to the temperature field before advecting
with this scheme, since ocean temperatures can be negative (or we
could use the Kelvin scale instead).

Recall that the advection of a tracer $C$ has an equation of the form
\begin{equation}
  \frac{\partial}{\partial t} \frac{ H_z C}{mn} =
  - \frac{\partial}{\partial \xi} F^\xi
  - \frac{\partial}{\partial \eta} F^\eta
  - \frac{\partial}{\partial \sigma} F^\sigma  .
\end{equation}
where we have introduced the advective fluxes:
\begin{align}
   F^\xi &= \frac{H_z u C}{n} \\
   F^\eta &= \frac{H_z v C}{m} \\
   F^\sigma &= \frac{H_z \Omega C}{mn} .
\end{align}
Rather than computing the centered average for these terms, 
we use the ``upwind'' or ``donor cell'' value for $C$. For instance:
\begin{equation}
   F^\xi_{i,j,k} = \frac{\overline{H_z}^\xi}{\overline{n}^\xi}
   \left[ \max(0,u_{i,j,k}) C_{i-1,j,k} +
   \min(0,u_{i,j,k}) C_{i,j,k} \right] .
\label{equp}
\end{equation}
This picks up the value of $C$ to the left if $u$ is positive,
otherwise it picks up the value to the right ($F^\xi$ is located at a
$u$ point on the grid).
   
Using these fluxes, we compute an upwind estimate of the new
tracer:
\begin{equation}
   C^{\star} = C^n - \frac{\Delta t\, mn}{H_z} \left(
   \frac{\partial F^\xi}{\partial \xi} +
   \frac{\partial F^\eta}{\partial \eta} +
   \frac{\partial F^\sigma}{\partial \sigma} \right)
\end{equation}
This estimate is used to create an ``antidiffusive'' velocity:
\begin{align}
   \tilde{u} &= \frac{1}{2}
      \left( \frac{|u|}{m} - \Delta t u^2 \right)
      \frac{m}{C^\star} \frac{\partial C^\star}{\partial \xi} -
      \frac{1}{2} \Delta t u \left( v \frac{n}{C^\star} 
      \frac{\partial C^\star}{\partial \eta} +
      w \frac{1}{C^\star} \frac{\partial C^\star}{\partial z}
      \right) \label{antiu} \\
   \tilde{v} &= \frac{1}{2}
      \left( \frac{|v|}{n} - \Delta t v^2 \right)
      \frac{n}{C^\star} \frac{\partial C^\star}{\partial \eta} -
      \frac{1}{2} \Delta t v \left( u \frac{m}{C^\star} 
      \frac{\partial C^\star}{\partial \xi} +
      w \frac{1}{C^\star} \frac{\partial C^\star}{\partial z}
      \right) \label{antiv} \\
   \tilde{w} &= \frac{1}{2}
      \left( |w| \Delta z - \Delta t w^2 \right)
      \frac{1}{C^\star} \frac{\partial C^\star}{\partial z} -
      \frac{1}{2} \Delta t w \left( u \frac{m}{C^\star} 
      \frac{\partial C^\star}{\partial \xi} +
      v \frac{n}{C^\star} \frac{\partial C^\star}{\partial \eta}
      \right) \label{antiw}
\end{align}
In SCRUM, these velocities are set to zero if the first term is smaller
than the second term (there is no such check in the ice MPDATA code,
nor is there mention of such a thing in the Smolarkiewicz papers).
Using these velocities in another upwind iteration leads to a second
estimate of the new tracer fields:
\begin{equation}
   C^{n+1} = C^\star - \frac{\Delta t\, mn}{H_z} \left(
   \frac{\partial F^\xi(\tilde{u})}{\partial \xi} +
   \frac{\partial F^\eta(\tilde{v})}{\partial \eta} +
   \frac{\partial F^\sigma(\tilde{w})}{\partial \sigma} \right)
\label{antidiff}
\end{equation}
In fact, this correction step can be repeated any number of times in
order to improve the solution.

Thus far, we have described the original MPDATA scheme which
maintains the positive definite nature of the tracer.  However, it does
not prevent under- and overshoots in a tracer such as salinity, in which
the initial range would be say 34 to 36 PSU. An optional nonoscillatory
modification to MPDATA is described in \cite{Smolark90}.
It uses ideas from the FCT (flux-corrected transport) scheme to limit
the antidiffusive fluxes. This modification is based on multiplying the
antidiffusive fluxes by a coefficient $C^\aster$ such that $0 \leq
C^\aster \leq 1$
and the resulting tracer value is bounded by $C^{\min} \leq
C^{n+1} \leq C^{\max}$.
  
First, we need to define $C^{\min}$ and $C^{\max}$.
\begin{align*}
  C^{\min}_{i,j,k} = \min ( &C_{i,j,k}, C_{i-1,j,k},
  C_{i+1,j,k},
  C_{i,j-1,k}, C_{i,j+1,k}, C_{i,j,k-1}, C_{i,j,k+1}, \\ 
  &C_{i,j,k}^\star, C_{i-1,j,k}^\star, C_{i+1,j,k}^\star,
  C_{i,j-1,k}^\star, C_{i,j+1,k}^\star, C_{i,j,k-1}^\star,
  C_{i,j,k+1}^\star ) \\
  C^{\max}_{i,j,k} = \max ( &C_{i,j,k}, C_{i-1,j,k},
  C_{i+1,j,k},
  C_{i,j-1,k}, C_{i,j+1,k}, C_{i,j,k-1}, C_{i,j,k+1}, \\
  &C_{i,j,k}^\star, C_{i-1,j,k}^\star, C_{i+1,j,k}^\star,
  C_{i,j-1,k}^\star, C_{i,j+1,k}^\star, C_{i,j,k-1}^\star,
  C_{i,j,k+1}^\star  )
\end{align*}
Then, we define the $\beta$-ratios:
\begin{align*}
  \left(\beta\! \uparrow \right)^{-1}
  \left(C^{\max}_{i,j,k} - C^\star_{i,j,k} \right) &=
  m \,\Delta t \left[ \max (0,\tilde{u}_{i,j,k}) C^\star_{i-1,j,k} - 
         \min (0,\tilde{u}_{i+1,j,k}) C^\star_{i+1,j,k} \right]
	 \\ &+
  n \,\Delta t \left[ \max (0,\tilde{v}_{i,j,k}) C^\star_{i,j-1,k} - 
         \min (0,\tilde{v}_{i,j+1,k}) C^\star_{i,j+1,k} \right]
	 \\ &+
  \frac{\Delta t}{\Delta z}
	     \left[ \max (0,\tilde{w}_{i,j,k-1}) C^\star_{i,j,k-1} - 
         \min (0,\tilde{w}_{i,j,k}) C^\star_{i,j,k+1} \right]  \\
  \left(\beta\! \downarrow \right)^{-1}
  \left(C^\star_{i,j,k} - C^{\min}_{i,j,k} \right) &=
  m \,\Delta t \left[ \max (0,\tilde{u}_{i+1,j,k}) C^\star_{i,j,k} - 
         \min (0,\tilde{u}_{i,j,k}) C^\star_{i,j,k} \right]
	 \\ &+
  n \,\Delta t \left[ \max (0,\tilde{v}_{i,j+1,k}) C^\star_{i,j,k} - 
         \min (0,\tilde{v}_{i,j,k}) C^\star_{i,j,k} \right]
	 \\ &+
  \frac{\Delta t}{\Delta z}
	     \left[ \max (0,\tilde{w}_{i,j,k}) C^\star_{i,j,k} - 
         \min (0,\tilde{w}_{i,j,k-1}) C^\star_{i,j,k} \right] 
\end{align*}
Finally, the flux-limited form of the antidiffusion velocities is:
\begin{align}
  \tilde{u}' &= \min(1, \beta\! \downarrow_{i-1,j,k},
      \beta\! \uparrow_{i,j,k}) \max(0, \tilde{u}) +
                \min(1, \beta\! \uparrow_{i-1,j,k},
		\beta\! \downarrow_{i,j,k}) \min(0, \tilde{u})  \\
  \tilde{v}' &= \min(1, \beta\! \downarrow_{i,j-1,k},
      \beta\! \uparrow_{i,j,k}) \max(0, \tilde{v}) +
                \min(1, \beta\! \uparrow_{i,j-1,k},
		\beta\! \downarrow_{i,j,k}) \min(0, \tilde{v})  \\
  \tilde{w}' &= \min(1, \beta\! \downarrow_{i,j,k},
      \beta\! \uparrow_{i,j,k+1}) \max(0, \tilde{w}) +
                \min(1, \beta\! \uparrow_{i,j,k},
		\beta\! \downarrow_{i,j,k+1}) \min(0, \tilde{w})
\end{align}
These velocities are used in equation (\ref{antidiff}) to produce the
flux-limited estimate of $C^{n+1}$.

