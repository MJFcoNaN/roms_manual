\section{Plotting Programs for Postprocessing}
\label{Plothist}
Hernan Arango has provided ROMS with some programs for creating plots
from the NetCDF history and restart files. Some example plots are shown
in \S\ref{Wave}. There are four plotting programs:
\begin{klist}
   \kitem{cnt} creates black-and-white plots of the horizontal fields,
including constant depth plots of the 3-D fields.
   \kitem{ccnt} creates color plots of the horizontal fields,
including constant depth plots of the 3-D fields.
   \kitem{sec} creates black-and-white plots of vertical slices through
the 3-D fields.  It includes on option of finding equal-spaced points
along a straight line through the curvilinear grid.
   \kitem{csec} creates color plots of vertical slices through
the 3-D fields.  It includes on option of finding equal-spaced points
along a straight line through the curvilinear grid.
\end{klist}
All of these program come with example input files.  For instance, the
input file for \code{ccnt} is called \code{ccnt.in} and is as follows:
\begin{verbatim}
1996 -1  :  year and starting year-day (use yearday<0, for no time label)
SCRUM 3.0
Coarse Arctic ocean with Budgell ice dynamics
ice thermodynamics

8     NFIELDS: number of fields to plot. Line below, field(s) types:
42,45,46,47,48,49,50,121,122,123,124  field identification:
FLDID(1:NFIELDS)
1     NLEVELS: number of levels and/or depths to plot (0 for all levels)
20           levels (>0) or depths (<0) to plot: FLDLEV(1:NLEVELS)
2     NFIELDS: number of fields to plot. Line below, field(s) types:
1,2            field identification: FLDID(1:NFIELDS)
1     NLEVELS: number of levels and/or depths to plot (0 for all levels)
1,2,3,4,5      levels (>0) or depths (<0) to plot: FLDLEV(1:NLEVELS)
0     FRSTD  : first day to plot
0     LASTD  : last day to plot
0     DSKIP  : plot every other DSKIP days (0.0 plot at its own time frequency)
0     VINTRP : vertical interpolation scheme: 0=linear, 1:cubic splines
0.0   PMIN   : field minimum value for color palette (0.0 for default)
0.0   PMAX   : field maximum value for color palette (0.0 for default)
1     ICNT   : draw contours between color bands: 0=no, 1=yes
0.0   ISOVAL : iso-surface value to process (see below)
1.2   VLWD   : vector line width (1.0 for default)
2.0   VLSCL  : vector length scale (1.0 for default)
1     IVINC  : vector grid sampling in the X-direction (1 for default)
1     JVINC  : vector grid sampling in the Y-direction (1 for default)
0     IREF   : secondary or reference field option (see below)
25    IDOVER : overlay field identification (for IREF=1,2 only)
1     LEVOVER: level of the overlay field (set to 0 if same as current FLDLEV)
0.0   RMIN   : overlay field minimum value to consider (0.0 for default)
0.0   RMAX   : overlay field maximum value to consider (0.0 for default)
10.0   LGRID : Desired longitude/latitude grid spacing (degrees)
4     IPROJ  : map projection (see below).
-60.0   PLON : projection Pole longitude (west values are negative).
90.0   PLAT  : projection Pole latitude (south values are negative).
0.0   ROTA   : projection rotation angle (clockwise; degrees).
0     LMSK   : flag to color mask land: [0] no, [1] yes
-1     NPAGE : number of plots per page (currently 1, 2, or 4)  
T     READGRD: logical switch to read in positions from grid NetCDF file.
F     PLTLOGO: logical switch draw Logo.
T     WRTHDR : logical switch to write out the plot header titles.
T     WRTBLAB: logical switch to write out the plot bottom title.
T     WRTRANG: logical switch to write out data range values and CI.

T     WRTFNAM: logical switch to write out input primary filename.
T     WRTDATE: logical switch to write out current date.
T     CST    : logical switch to read and plot coastlines and islands.
50.0 50.0    : bottom and top map latitudes (south values are negative).
-110.0 80.0  : left and right map longitudes (west values are negative).
/d2/kate/plot/varid.dat
/d1/arango/scrum3.0/plot/Palettes/gs1.pal
/d2/kate/plot/default.cnt
ice_rst.nc
scrum_rst_1.nc
/d2/kate/arctic/gridpak/arctic_grid_2.nc
/u1/coasts/coast.dat


c
c=======================================================================
c  Copyright (c) 1996 Rutgers University                             ===
c=======================================================================
c

 *** Above FILENAMES:

            1st line: input; variables ID file.
            2nd line: input; color palette file.
            3rd line: input; contour parameters.
            4th line: input; primary NetCDF file.
            5th line: input; secondary NetCDF file.
            6th line: input; grid NetCDF file.
            7th line: input; coastlines file.

 *** IREF:  Secondary or reference field option:
           -1   Overlay horizontal grid
            0   no secondary or reference field to plot
            1   plot field overlay from primary file
            2   plot field overlay from secondary file
            3   primary - secondary file (field subtraction)
            4   Day0 - DayN (field subtraction)

 *** IPROJ: Map Projections option.  Some of the values for the
            projection Pole and rotation angle are suggested.
            Check the NCAR manual for details.

           [1] Cylindrical equidistant: PLON=0, PLAT=0, ROTA=0
           [2] Mercator: PLON=?, PLAT=0, ROTA=0
           [3] Lambert conformal conic: PLON=?, PLAT=?, ROTA=?
           [4] Stereographic azimuthal:  PLON=?, PLAT=90 or -90, ROTA=0

 *** IVINC, JVINC: vector grid sampling.  If either value is negative,
                   the velocity vectors at drawn at PSI-points.  Otherwise,
                   if both values are positive, the vectors are drawn at
                   interior RHO-points.

 *** LGRID: Longitude/latitude grid spacing.  The grid is drawn at
            LGRID spacing starting from specified map lower corner.
            If LGRID is negative, the latitude labels in the map are
            rotated 90 degrees to avoid label congestion, if any.

 *** NPAGE: Number of plots per page.  Set this parameter to a negative
            value (-1, -2, or -4) to activate preservation of the plot
            aspect ratio.


  Plotting Fields classification: (* derived fields)

 [  1]  IDUTOT  total velocity component in the XI-direction (cm/s).
 [  2]  IDVTOT  total velocity component in the ETA-direction (cm/s).
*[  3]  IDTVEC  total velocity vectors (cm/s).
*[  4]  IDTMAG  total velocity vector magnitude (cm/s).
                 :
                 :
\end{verbatim}
As you can see, there are comments describing what needs to be
done.  Please see the variable ID file for the complete list of fields
which can be plotted---this list changes as Hernan adds the ability to
plot new fields. Also, check your default.cnt file for other vector and
contour parameters. The palette file includes two number systems, one
in the scale from 0 to 255 and the other from 0 to 1.
The ROMS plotting uses the first while the SEOM plotting uses the
second.
