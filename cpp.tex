\section{The C preprocessor}
\label{Cpp}

The C preprocessor, \code{cpp}, is a standalone program which comes
with most C compilers. On older UNIX systems it was not in the default
path, but in \code{/lib} or in \code{/usr/lib}. Most modern systems
have some version of \code{gcc} which comes with a quite capable
\code{cpp}. Just be sure to give it the \code{-traditional} flag.

This Appendix describes the C preprocessor as used in ROMS with the
Fortran language.  A more complete description is given by
\citet{K&R}.  More practical advice on using
\code{cpp} is given by \citet{Hazard}.

\subsection{File inclusion}
Placing common blocks in smaller files, which are then included in each
subroutine, is the easiest way to make sure that the common blocks are
declared consistently.  Many Fortran compilers support an include
statement where the compiler replaces the line
\begin{verbatim}
            include 'file.h'
\end{verbatim}
with the contents of \code{file.h}; \code{file.h} is assumed
to be in the current directory.  The C preprocessor has an equivalent
include statement:
\begin{verbatim}
      #include "file.h"
\end{verbatim}
We are using the C preprocessor style of include because many of the
ROMS include files are not pure Fortran and must be processed by
\code{cpp}.

\subsection{Macro substitution}
A macro definition has the form
\begin{verbatim}
      #define    name           replacement text
\end{verbatim}
where ``name'' would be replaced with ``replacement text''
throughout the rest of the file.  This was used in SCRUM as a
reasonably portable way to get 64-bit precision:
\begin{verbatim}
      #define  BIGREAL     real*8
\end{verbatim}
It is customary to use uppercase for \code{cpp} macros---the C
preprocessor is case sensitive.

It is also possible to define macros with arguments, as in
\begin{verbatim}
      #define av2(a1,a2)        (.5 * ((a1) + (a2)))
\end{verbatim}
although this is riskier than the equivalent statement function
\begin{verbatim}
            av2(a1,a2) = .5 * (a1 + a2)
\end{verbatim}
The statement function is preferable because it allows the compiler
to do type checking and because you don't have to be as careful
about using enough parentheses.

The third form of macro has no replacement text at all:
\begin{verbatim}
      #define  MASKING
\end{verbatim}
In this case, \code{MASKING} will evaluate to \code{true} in the
conditional tests described below.

\subsection{Conditional inclusion}
It is possible to control which parts of the code are seen by the
Fortran compiler by the use of \code{cpp}'s conditional inclusion.
For example, the statements
\begin{verbatim}
      #ifdef MASKING
      # include "mask.h"
      #endif /* MASKING */
              :
      # ifdef MASKING
      c
      c  Apply Land/Sea mask: slipperiness.
      c
            do j=1,M
              do i=2,Lm
                Uflux(i,j)=Uflux(i,j)*pmask(i,j)
              enddo
            enddo
      # endif /* MASKING */
\end{verbatim}
will not be in the Fortran source code if \code{MASKING} has not
been defined.  Likewise,
\code{\#ifndef} tests for a macro being undefined:
\begin{verbatim}
      #ifndef RMDOCINC
      c  rmask       Mask at RHO-points (0=Land, 1=Sea).
      c  pmask       Slipperiness mask at PSI-points (0=Land, 1=Sea,
      c                                               1-gamma2=boundary).
      c  umask       Mask at U-points (0=Land, 1=Sea).
      c  vmask       Mask at V-points (0=Land, 1=Sea).
      c
      c=======================================================================
      #endif
\end{verbatim}

There are also \code{\#else} and \code{\#elif} (else if) statements.
%although \code{\#elif} is newer and is not supported by all versions of
%\code{cpp}.
An example using \code{\#else} and \code{\#elif} is
shown:
\begin{verbatim}
#ifdef SOLITON
        real(r8) :: g = 1.0_r8                  ! non-dimensional
# elif defined WBC_1 || defined WBC_2 || defined WBC_3
        real(r8) :: g = 9.8_r8                  ! m/s2
# elif defined CIRCLE
        real(r8) :: g = 3.92e-2_r8                  ! m/s2
#else
        real(r8) :: g = 9.81_r8                 ! m/s2
#endif
\end{verbatim}

Actually, \code{\#ifdef} is a restricted version of the more general
test
\begin{verbatim}
      #if     expression
\end{verbatim}
where ``expression'' is a constant integer value.  Zero evaluates
to \code{false} and everything else is considered \code{true}.
Compound expressions may be built using the C logical operators:
\begin{verbatim}
        &&           logical and
        ||           logical or
        !            logical not
\end{verbatim}
These symbols would be used as in the following example:
\begin{verbatim}
      #if defined CANYON_A || defined CANYON_B
            do j=0,M
              do i=0,L
                yc=c32000-c16000*(sin(pi*xr(i,j)/xl))**24
                h(i,j)=c20+p5*(hmax-c20)*(c1+tanh((yr(i,j)-yc)/c10000))
              enddo
            enddo
      #endif
\end{verbatim}

\subsection{C comments}
The C preprocessor will also delete C language comments starting
with \code{/*} and ending with \code{*/} as in:
\begin{verbatim}
       #endif  /* MASKING */
\end{verbatim}
When mixed with Fortran code, it is safer to use a Fortran
comment.

\subsection{A note on style}
In ROMS, the style adopted is as shown above and here:
\begin{verbatim}
#define NEP5
#if defined NEP5 || defined BERING
        :
# ifdef MASKING
        :
# endif
#endif
\end{verbatim}
The \code{gnu} community has instead adopted this style:
\begin{verbatim}
#define NEP5  1
#if NEP5 || BERING
        :
# if MASKING
        :
# endif
#endif
\end{verbatim}
Both styles work and you should be aware that there's more than one
way to do it.

\subsection{Potential problems}
The use of the C preprocessor is not entirely free of problems, but
many can be worked around or avoided by using the \code{gnu} version of
\code{cpp} with the \code{-traditional} flag.
   \begin{enumerate}
     \item Apostrophes in Fortran comments.  \code{cpp} does not
     know that it is in a comment and some versions will complain about
     unmatched apostrophes in the following:
     \begin{verbatim}
  !  Some useful comment about Green's functions.
     \end{verbatim}
     \item \CC\ comments.  Some of the newer versions of
     \code{cpp} will remove \CC\ comments which go from '//' to
     the end of the line.  Some perfectly reasonable Fortran lines
     contain two consecutive slashes, such as:
     \begin{verbatim}
        common // var1, var2
    44  format(//)
     \end{verbatim}
     and Fortran 90 string concatenation:
     \begin{verbatim}
        mystring = 'Hello, ' // 'World!'
     \end{verbatim}
     \item Macro replacement.  One feature of \code{cpp} is that you
     can define macros and have it perform replacements.  The code:
     \begin{verbatim}
  #define REAL double precision
        REAL really_long_variable, second_long_variable
     \end{verbatim}
     becomes
     \begin{verbatim}
        double precision really_long_variable, second_long_variable
     \end{verbatim}
     and you run the risk of creating lines which are longer than 72
     characters in length.

     Also, make sure that your macros will not be found anywhere else
     in the code.  I used to use \code{\#define DOUBLE} for double
     precision until it was pointed out to me that \code{DOUBLE}
     \code{PRECISION} is perfectly valid Fortran.  Since a string
     that is simply ``defined'' becomes 1, the macro processor would
     turn this into \code{1 PRECISION}.
  \end{enumerate}

\subsection{Modern Fortran}
I started working on these ocean models before 1990, much less before
Fortran 90 compilers were generally available.  Fortran 90's \code{kind}
feature as used in ROMS is a better way to handle the \code{BIGREAL}
type declarations.  On the other hand, Fortran 90 does not include
conditional compilation.  However, it is deemed useful enough that the
Fortran 2003 standard has the coco means of conditional compilation.
We {\em might} start using this in about ten years.
